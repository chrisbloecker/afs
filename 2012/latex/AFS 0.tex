\documentclass[]{beamer}
\usepackage[utf8x]{inputenc}
\usepackage{beamerthemeshadow}
\usepackage{ucs}
\usepackage{amsmath}
\usepackage{amsfonts}
\usepackage{amssymb}
\usepackage[ngerman]{babel}
\usepackage{graphicx}
\usepackage{textpos}
\usepackage{csquotes}
\usepackage{url}
\usepackage{color}
\usepackage{tikz}
\usetikzlibrary{shapes,backgrounds}

\usetheme[compress]{Berlin}
\usefonttheme{default}
\useoutertheme{infolines}

\setbeamertemplate{footline}%{infolines theme}
{
\leavevmode%
\hbox{%
\begin{beamercolorbox}[wd=.333333\paperwidth,ht=2.25ex,dp=1ex,center]{author in head/foot}%
\usebeamerfont{author in head/foot}\insertshortauthor%~~(\insertshortinstitute)
\end{beamercolorbox}%
\begin{beamercolorbox}[wd=.333333\paperwidth,ht=2.25ex,dp=1ex,center]{title in head/foot}%
\usebeamerfont{title in head/foot}\insertshorttitle
\end{beamercolorbox}%
\begin{beamercolorbox}[wd=.333333\paperwidth,ht=2.25ex,dp=1ex,center]{date in head/foot}%
\usebeamerfont{date in head/foot}\insertshortdate{}
%\insertframenumber{} / \inserttotalframenumber\hspace*{2ex} % ohne Seitenzahlen
\end{beamercolorbox}}%
\vskip0pt%
}

\setbeamertemplate{frametitle}
{%
\vbox{}\vskip-3.9ex
\begin{beamercolorbox}[wd=\paperwidth,ht=2.0ex,leftskip=0.4cm,dp=0.6ex,]{frametitle}
\usebeamerfont{header_font_title}{\insertframetitle}
\end{beamercolorbox}%
}

% Schusterjunge & Hurenkind
\clubpenalty = 10000
\widowpenalty = 10000
\displaywidowpenalty = 10000
\hbadness = 10000

\author[Christopher Blöcker, B. Sc.]{Christopher Blöcker, B. Sc.\\ inf9900@fh-wedel.de}
\title[AFS Tutorium]{Tutorium\\Automaten und Formale Sprachen}
\date{SS 2012}
\subtitle{Teil 0.}
\subject{Automaten und Formale Sprachen}
\begin{document}

\begin{frame}{}
  \titlepage
\end{frame}



\section{Vollständige Induktion}

\subsection{Prinzip}

\begin{frame}[<+->][squeeze]{}
  \begin{alertblock}{Aufgabe}
    Es sei eine Aussage der Form
    \[\bigwedge_{x \in \mathbb{N}} h \left( x \right)\]
    zu beweisen, dabei sei $h$ ein Prädikat.
  \end{alertblock}
  
  \begin{block}{Satz}
    \[\underbrace{h \left( 0 \right)}_{\text{Verankerung}}
      \land
      \underbrace{\bigwedge_{n \in \mathbb{N}} \left[h \left( n \right) \rightarrow h \left( \sigma \left( n \right) \right)\right]}_{\text{Schritt}} 
      \rightarrow 
      \underbrace{\bigwedge_{n \in \mathbb{N}} h \left( n \right)}_{Schluss}\] 
    Dabei bezeichne $\sigma \left( n \right)$ den Nachfolger von $n$.
  \end{block}
\end{frame}

\begin{frame}[squeeze]{}
  \begin{block}{Satz}
    \[\underbrace{h \left( 0 \right)}_{\text{Verankerung}}
      \land
      \underbrace{\bigwedge_{n \in \mathbb{N}} \left[h \left( n \right) \rightarrow h \left( \sigma \left( n \right) \right)\right]}_{\text{Schritt}} 
      \rightarrow 
      \underbrace{\bigwedge_{n \in \mathbb{N}} h \left( n \right)}_{Schluss}\] 
    Dabei bezeichne $\sigma \left( n \right)$ den Nachfolger von $n$.
  \end{block}
  
  \begin{block}{Satz (in Worten)}
    Sei $h \left( x \right)$ eine Aussageform über der Menge der natürlichen Zahlen. \\ 
    \vspace*{1em}
    Wenn $h \left( 0 \right)$ wahr ist \hfill \textit{Verankerung} \\
    \vspace*{0.5em}
    \pause
    \textbf{und} \\
    \vspace*{0.5em}
    wenn für jedes beliebige $n \in \mathbb{N}$ aus der Wahrheit von $h \left( n \right)$ \\ stets die Wahrheit von $h \left( \sigma \left( n \right) \right)$ folgt, \hfill \textit{Schritt} \\
    \vspace*{0.5em}
    \pause
    \textbf{dann} \\
    \vspace*{0.5em}
    gilt die Aussage für alle natürlichen Zahlen. \hfill \textit{Schluss}
  \end{block}
\end{frame}

\subsection{\textsc{Peano}'sches Axiomensystem}
\begin{frame}[squeeze]{}
  \begin{block}{Beispiel: \textsc{Peano}'sches Axiomensystem}
    Axiomatische Definition der Menge der Natürlichen Zahlen $\mathbb{N}$.
    \begin{enumerate}
      \item \textbf{Verankerung}: Es gelte $0 \in \mathbb{N}$
      \vspace*{1em}
      \item $\sigma \colon \mathbb{N} \to \mathbb{N}$ sei eine Funktion, die jedem $n \in \mathbb{N}$ einen (eindeutigen!) Nachfolger zuordnet
      \vspace*{1em}
      \item $\sigma$ sei injektiv, also $\sigma \left( m \right) = \sigma \left( n \right) \to m = n$
      \vspace*{1em}
      \item $0$ ist Nachfolger keiner natürlichen Zahl, d.h. $\underset{n \in \mathbb{N}}{\bigwedge} \sigma \left( n \right) \neq 0$
      \vspace*{1em}
      \item \textbf{Induktionsaxiom}: Jedes $n \in \mathbb{N}$ kann durch endlich häufige Anwendung von $\sigma$ auf $0$ erzeugt werden $n = \underbrace{\left( \sigma \circ \ldots \circ \sigma \right)}_{n \text{ mal}} \left( 0 \right)$
    \end{enumerate}
  \end{block}
\end{frame}



\section{Mengenlehre}

\begin{frame}[<+->][squeeze]{}
  \begin{block}{Mengendefinition nach \textsc{Cantor}}
    Eine Menge ist eine Zusammenfassung bestimmter wohlunterschiedener
    Objekte unserer Anschauung oder unseres Denkens zu einem Ganzen. Von jedem
    dieser Objekte muss eindeutig feststehen, ob es zur Menge gehört oder nicht.
    Die zur Menge gehörenden Objekte nennt man die Elemente der Menge.
  \end{block}
  
  \begin{alertblock}{Notwendig}
    Wir fordern, dass die Elemente auf Gleichheit getestet werden können.
  \end{alertblock}  
  
  \begin{block}{Mengenoperationen}
    \[\begin{array}{l c l}
        \mathcal{A} \subseteq \mathcal{B} & \Leftrightarrow & \underset{a \in \mathcal{A}}{\bigwedge} a \in \mathcal{B} \vspace*{0.25em}\\
        \mathcal{A} \cap      \mathcal{B} & \Leftrightarrow & \{ \ x \ | \ x \in \mathcal{A} \ \land \ x \in    \mathcal{B} \ \} \\
        \mathcal{A} \cup      \mathcal{B} & \Leftrightarrow & \{ \ x \ | \ x \in \mathcal{A} \ \vee   \ x \in    \mathcal{B} \ \} \\
        \mathcal{A} \setminus \mathcal{B} & \Leftrightarrow & \{ \ x \ | \ x \in \mathcal{A} \ \land \ x \notin \mathcal{B} \ \} \\
        \mathcal{A} \triangle \mathcal{B} & \Leftrightarrow & \left( \mathcal{A} \cup \mathcal{B} \right) \setminus \left( \mathcal{A} \cap \mathcal{B} \right) \\
      \end{array}\]
  \end{block}
\end{frame}

\begin{frame}[squeeze]{}
  \def\firstcircle{(0,0) circle (1.5cm)}
  \def\secondcircle{(0:2cm) circle (1.5cm)}
  \def\thirdcircle{(6,0) circle (1.5cm)}
  \def\fourthcircle{(8,0) circle (1.5cm)}

  \colorlet{circle edge}{blue!50}
  \colorlet{circle area}{blue!20}

  \tikzset{filled/.style={ fill=circle area, draw=circle edge, thick}
                         , outline/.style={draw=circle edge, thick}
                         }

  \setlength{\parskip}{5mm}
  
  \vspace*{-1em}
  \begin{center}
  % Set A and B
  \begin{tikzpicture}
    \begin{scope}
        \clip \firstcircle;
        \fill[filled] \secondcircle;
    \end{scope}
    \draw[outline] \firstcircle node {$\mathcal{A}$};
    \draw[outline] \secondcircle node {$\mathcal{B}$};
    \node[anchor=south] at (current bounding box.north) {$\mathcal{A} \cap \mathcal{B}$};
    
    
    \draw[filled] \thirdcircle node {$\mathcal{A}$}
                  \fourthcircle node {$\mathcal{B}$};
    \node[anchor=south] at (7,1.5) {$\mathcal{A} \cup \mathcal{B}$};
  \end{tikzpicture}

  % Set A but not B
  \begin{tikzpicture}
    \begin{scope}
        \clip \firstcircle;
        \draw[filled, even odd rule] \firstcircle node {$\mathcal{A}$}
                                     \secondcircle;
    \end{scope}
    \draw[outline] \firstcircle
                   \secondcircle node {$\mathcal{B}$};
    \node[anchor=south] at (current bounding box.north) {$\mathcal{A} \setminus \mathcal{B}$};
    
    
    \draw[filled, even odd rule] \thirdcircle node {$\mathcal{A}$}
                                 \fourthcircle node{$\mathcal{B}$};
    \node[anchor=south] at (7,1.5) {$\mathcal{A} \Delta \mathcal{B}$};
  \end{tikzpicture}
  \end{center}
\end{frame}


\begin{frame}[<+->][squeeze]{}
  \begin{alertblock}{Frage}
    Es existiere ein Dorf, in dem es verboten sei, einen Bart zu tragen und
    \begin{enumerate}
      \item<0-> der Dorfbarbier denjenigen Männern den Bart schneide, die sich \textbf{nicht} selbst den Bart schneiden und
      \item<0-> der Dorfbarbier der einzige Mann sei.
    \end{enumerate}
    $\mathcal{M}$ sei die Menge der Männer, denen der Dorfbarbier den Bart schneidet, \\ $x$ sei der Dorfbarbier. \\
    \[x \overset{?}{\in} \mathcal{M} \text{ bzw } \mathcal{M} \overset{?}{=} \emptyset\]
  \end{alertblock}
  
  \begin{exampleblock}{Antwort}
    Die Frage ist unentscheidbar, denn
    \[x \in \mathcal{M} \Leftrightarrow x \notin \mathcal{M}.\]
  \end{exampleblock}
\end{frame}

\section{Relationen}
\begin{frame}[<+->][squeeze]{}
  \begin{definition}
    Eine ($2$-stellige) Relation zwischen den Mengen $\mathcal{M}$ und $\mathcal{N}$ ist eine
Teilmenge $\mathcal{R}$ der Produktmenge $\mathcal{M} \times \mathcal{N}$.
    \[\mathcal{R} \subseteq \mathcal{M} \times \mathcal{N}.\]
    Dabei steht $m \in \mathcal{M}$ mit $n \in \mathcal{N}$ in Relation, wenn
    \[(m, n) \in \mathcal{R}.\]
    Man schreibt dafür auch
    \[m \overset{\mathcal{R}}{\sim} n \text{ oder } m\mathcal{R}n.\]
  \end{definition}
\end{frame}

\subsection{Äquivalenzrelation}
\begin{frame}[<+->][squeeze]{}
  \begin{block}{Äquivalenzrelation}
    Sei $\mathcal{R} \subseteq \mathcal{M}^2$. $\mathcal{R}$ definiert eine Äquivalenzrelation, wenn gilt
    \begin{enumerate}
      \item Reflexivität : \hspace{8pt} $\underset{x \in \mathcal{M}}{\bigwedge} x \overset{\mathcal{R}}{\sim} x$
      \item Symmetrie : \hspace{7pt} $\underset{x, y \in \mathcal{M}}{\bigwedge} (x \overset{\mathcal{R}}{\sim} y) \rightarrow (y \overset{\mathcal{R}}{\sim} x)$
      \item Transitivität : $\underset{x, y, z \in \mathcal{M}}{\bigwedge} [(x \overset{\mathcal{R}}{\sim} y) \land (y \overset{\mathcal{R}}{\sim} z)] \rightarrow (x \overset{\mathcal{R}}{\sim} z)$
    \end{enumerate}
  \end{block}
\end{frame}

\subsection{Ordnungsrelation}
\begin{frame}[squeeze]{}
  \vspace*{-0.25em}
  \begin{block}{Ordnungsrelation}
    Sei $\mathcal{R} \subseteq \mathcal{M}^2$. $\mathcal{R}$ definiert eine Halbordnung, wenn gilt
    \pause
    \begin{enumerate}
      \item \hspace*{4pt}Reflexivität : \hspace{15pt} $\underset{x \in \mathcal{M}}{\bigwedge} x \overset{\mathcal{R}}{\sim} x$ \\
        \hspace*{-22pt}\textbf{oder} Irreflexivität : \hspace{12pt} $\underset{x \in \mathcal{M}}{\bigwedge} x \overset{\mathcal{R}}{\nsim} x$
        \pause
      \item \hspace*{4pt}Antisymmetrie : $\underset{x, y \in \mathcal{M}}{\bigwedge} [(x \overset{\mathcal{R}}{\sim} y) \land (y \overset{\mathcal{R}}{\sim} x)] \rightarrow (x = y)$ \\
        \hspace*{-22pt}\textbf{oder} Asymmetrie : \hspace{9pt} $\underset{x, y \in \mathcal{M}}{\bigwedge} (x \overset{\mathcal{R}}{\sim} y) \rightarrow (x \neq y)$
        \pause
      \item \hspace*{4pt}Transitivität : \hspace{4pt} $\underset{x, y, z \in \mathcal{M}}{\bigwedge} [(x \overset{\mathcal{R}}{\sim} y) \land (y \overset{\mathcal{R}}{\sim} z)] \rightarrow (x \overset{\mathcal{R}}{\sim} z)$
        \pause
    \end{enumerate}
  \end{block}
  \vspace*{-0.5em}
  \begin{definition}
    Ist zusätzlich die Eigenschaft der Linearität erfüllt, so definiert $\mathcal{R}$ eine Totalordnung.
    \begin{enumerate}
      \item Linearität : $\underset{x, y \in \mathcal{M}}{\bigwedge} (x \overset{\mathcal{R}}{\sim} y) \vee (y \overset{\mathcal{R}}{\sim} x)$
    \end{enumerate}
  \end{definition}
\end{frame}

\section{Algebraische Strukturen}
\begin{frame}[squeeze]{}
  \vspace*{-0.25em}
  \begin{block}{Algebraische Struktur}
    Eine algebraische Struktur ist ein Paar, bestehend aus einer Menge $\mathcal{M}$ und einer Familie von $n_i$-stelligen Verknüpfungen $f_i$ auf den Elementen der Menge mit $f_i \colon \mathcal{M}^{n_i} \to \mathcal{M}$.
    \[\left(\mathcal{M}, \left( f_i \right) \right)\]
  \end{block}
  \vspace*{-0.5em}
  
  \pause  
  
  \begin{block}{Gruppoid}
    Ein Gruppoid besteht aus einer Menge $\mathcal{M}$ und einer inneren, binären Verknüpfung $\ast \colon \mathcal{M} \times \mathcal{M} \to \mathcal{M}$.
    \begin{enumerate}
      \item $\left( \mathbb{Z}, - \right)$
      \item $\left( \mathbb{N}, \ast \left(a, b \right) = a^b \right)$
    \end{enumerate}
  \end{block}
  \vspace*{-0.5em}
  
  \pause  
  
  \begin{block}{Halbgruppe}
    Eine Halbgruppe besteht aus einer Menge $\mathcal{M}$ und einer inneren, binären, \textbf{assoziativen} Verknüpfung $\ast \colon \mathcal{M} \times \mathcal{M} \to \mathcal{M}$. \\
    Es muss gelten \[\bigwedge_{a, b, c \in \mathcal{M}} a \ast \left( b \ast c \right) = \left( a \ast b \right) \ast c\]
  \end{block}
\end{frame}

\begin{frame}[squeeze]{}
  \vspace*{-0.25em}
  \begin{block}{Monoid}
    Ein Monoid ist eine Halbgruppe mit ausgezeichnetem neutralen Element $e$: $(\mathcal{M}, *, e)$ mit \vspace*{-0.25em}
      \[\bigwedge_{a \in \mathcal{M}} e * a = a = a * e\]
  \end{block}
  \vspace*{-0.5em}
  \pause
  
  \begin{block}{Gruppe}
    Eine Gruppe ist ein Monoid mit einer zusätzlichen unären Operation $\text{}^{-1}\colon \mathcal{M} \to \mathcal{M}$, die zu jedem Element das inverse bezüglich $\ast$ bestimmt: $\left( \mathcal{M}, \ast, e, \text{}^{-1} \right)$. \vspace*{-0.25em}
    \[\bigwedge_{a \in \mathcal{M}} a \ast a^{-1} = e = a^{-1} \ast a\]
  \end{block}
  \vspace*{-0.5em}
  
  \pause
  
  \begin{block}{\textsc{Abel}'sche Gruppe}
    Eine \textsc{abel}'sche Gruppe ist eine Gruppe mit einer \textbf{kommutativen} Operation $\ast\colon \mathcal{M} \times \mathcal{M} \to \mathcal{M}$ mit \vspace*{-0.25em}
    \[\bigwedge_{a, b \in \mathcal{M}} a \ast b = b \ast a\]
  \end{block}
\end{frame}

\section{Zahlentheorie}
\subsection{kgV und ggT}
\begin{frame}[<+->][squeeze]{}
  \begin{definition}
    Seien $m, n$ zwei beliebige Zahlen in $\mathbb{N}$.
    \vspace*{0.5em}\\
    Die Zahl $d \in \mathbb{N}$ heisst \textbf{größter gemeinsamer Teiler} der Zahlen $m$ und $n$ (kurz: \textbf{ggT}), wenn gilt:
    \[\left( d \, | \, m \right) \land \left( d \, | \, n \right) \land \underset{t \in \mathbb{N}}{\bigwedge} \left[ \left( t \, | \, m \right) \land \left( t \, | \, n \right) \rightarrow \left( t \, | \, d \right) \right]\]
  \end{definition}
  
  \begin{definition}
    Seien $m, n$ zwei beliebige Zahlen in $\mathbb{N}$. \vspace*{0.5em}
    \\
    Die Zahl $k \in \mathbb{N}$ heisst \textbf{kleinstes gemeinsames Vielfaches} der Zahlen $m$ und $n$ (kurz: \textbf{kgV}), wenn gilt:
    \[\left( m \, | \, k \right) \land \left( n \, | \, k \right) \land \underset{s \in \mathbb{N}}{\bigwedge} \left[ \left( m \, | \, s \right) \land \left( n \, | \, s \right) \rightarrow \left( k \, | \, s \right) \right]\]
  \end{definition}
\end{frame}

\begin{frame}[squeeze]{}
  \vspace*{-0.25em}
  \begin{block}{Zusammenhang von kgV und ggT}
    Zwischen $\mathrm{ggT} \left( m, n \right)$ und $\mathrm{kgV} \left( m, n \right)$ zweier Zahlen $m, n \in \mathbb{N}$ besteht der folgende Zusammenhang: \vspace*{-0.5em}
    \[\mathrm{ggT} \left( m, n \right) \cdot \mathrm{kgV} \left( m, n \right) = m \cdot n\]
  \end{block}
  \vspace*{-0.5em}
  \pause
  
  \begin{exampleblock}{Berechnung}
    Der \textbf{ggT} zweier Zahlen lässt sich mit Hilfe des \textsc{Euklidischen Algorithmus} berechnen.
  \end{exampleblock}
  \vspace*{-0.5em}
  \pause
  
  \begin{block}{Berechnung von $\mathrm{ggT} \left( m, n \right)$ mit $m, n \in \mathbb{N}, m \geq n$}
    \[\begin{array}{ c c c c c}
      m &= &q_0 \cdot n &+ &r_0 \\
      \pause
      n &= &q_1 \cdot r_0 &+ &r_1 \\
      \pause
      \vdots & & \vdots & & \vdots \\
      r_{n-1} &= &q_n \cdot r_n &+ &0 \\
    \end{array}\]
    \pause
    Es gilt: $\mathrm{ggT} \left(m, n \right) = r_{n-1}$.
  \end{block}
\end{frame}

\begin{frame}[<+->][squeeze]{}
  \begin{exampleblock}{Euklidischer Algorithmus (C - Rekursiv)}
    \texttt{\textcolor{blue}{unsigned} ggt(\textcolor{blue}{unsigned} m, \textcolor{blue}{unsigned} n) \{ } \\
    \texttt{$\ \ $ if (m $<$ n) \hspace*{0.12em} return ggt(n, m);} \\
    \texttt{$\ \ $ if (n == 1) return 1;} \\
    \texttt{$\ \ $ if (n == 0) return m;} \\
    \texttt{$\ \ $ return ggt(n, m \% n);} \\
    \}
  \end{exampleblock}

  \begin{exampleblock}{Euklidischer Algorithmus (Haskell)}
    \texttt{module \textcolor{blue}{GGT} where} \\
    \vspace*{0.5em}
    \texttt{ggt \textbf{::} \textcolor{blue}{Integral} a => a -> a -> a} \\
    ggt m n \\
      \texttt{$\ \ |$ m $<$ n}   \hspace{22pt} \texttt{= ggt n m} \\
      \texttt{$\ \ |$ n == 1}    \hspace{19pt} \texttt{= 1} \\
      \texttt{$\ \ |$ n == 0}    \hspace{19pt} \texttt{= m} \\
      \texttt{$\ \ |$ otherwise} \hspace{2pt} \texttt{= ggt n (m `mod` n)}
  \end{exampleblock}
\end{frame}

\end{document}
