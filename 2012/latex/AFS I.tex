\documentclass[]{beamer}
\usepackage[utf8x]{inputenc}
\usepackage{beamerthemeshadow}
\usepackage{ucs}
\usepackage{amsmath}
\usepackage{amsfonts}
\usepackage{amssymb}
\usepackage[ngerman]{babel}
\usepackage{graphicx}
\usepackage{textpos}
\usepackage{csquotes}
\usepackage{url}
\usepackage{color}

\usetheme[compress]{Berlin}
\usefonttheme{default}
\useoutertheme{infolines}

\setbeamertemplate{footline}%{infolines theme}
{
\leavevmode%
\hbox{%
\begin{beamercolorbox}[wd=.333333\paperwidth,ht=2.25ex,dp=1ex,center]{author in head/foot}%
\usebeamerfont{author in head/foot}\insertshortauthor%~~(\insertshortinstitute)
\end{beamercolorbox}%
\begin{beamercolorbox}[wd=.333333\paperwidth,ht=2.25ex,dp=1ex,center]{title in head/foot}%
\usebeamerfont{title in head/foot}\insertshorttitle
\end{beamercolorbox}%
\begin{beamercolorbox}[wd=.333333\paperwidth,ht=2.25ex,dp=1ex,center]{date in head/foot}%
\usebeamerfont{date in head/foot}\insertshortdate{}
%\insertframenumber{} / \inserttotalframenumber\hspace*{2ex} % ohne Seitenzahlen
\end{beamercolorbox}}%
\vskip0pt%
}

\setbeamertemplate{frametitle}
{%
\vbox{}\vskip-3.9ex
\begin{beamercolorbox}[wd=\paperwidth,ht=2.0ex,leftskip=0.4cm,dp=0.6ex,]{frametitle}
\usebeamerfont{header_font_title}{\insertframetitle}
\end{beamercolorbox}%
}

% Schusterjunge & Hurenkind
\clubpenalty = 10000
\widowpenalty = 10000
\displaywidowpenalty = 10000
\hbadness = 10000

\author[Christopher Blöcker, B. Sc.]{Christopher Blöcker, B. Sc.\\ inf9900@fh-wedel.de}
\title[AFS Tutorium]{Tutorium\\Automaten und Formale Sprachen}
\date{SS 2012}
\subtitle{Teil I.}
\subject{Automaten und Formale Sprachen}
\begin{document}

\begin{frame}{}
  \titlepage
\end{frame}

\begin{frame}[squeeze]{}
  \begin{block}{JFLAP}
    Programm zur Simulation von Automaten etc., u.a.
    \begin{itemize}
      \item Endliche Automaten (+ Reguläre Ausdrücke)
      \item Kellerautomaten
      \item Turingmaschinen
      \item Grammatiken
      \item $\ldots$
    \end{itemize}
  \end{block}
  
  \begin{exampleblock}{Download}
    Google $\rightarrow$ "JFLAP download".
  \end{exampleblock}
  
  \begin{exampleblock}{Start}
    \begin{itemize}
      \item Linux : java -jar /pfad/JFLAP.jar $\&$
      \item Windows : durch Anklicken
    \end{itemize}
  \end{exampleblock}
\end{frame}

\section{Sprachen}
\begin{frame}[<+->][squeeze]{}
  \begin{alertblock}{Frage}
    Warum interessieren wir uns für Formale Sprachen?
  \end{alertblock}
  
  \begin{exampleblock}{Antwort}
    Die Theorie der Formalen Sprachen bildet die Grundlage für Programmiersprachen. \\
    \vspace*{0.5em}
    Wenn die syntaktische Beschreibung einer Programmiersprache gegeben ist, so können wir anhand dieser Information entscheiden, ob ein gegebenes Programm gültig ist und zu dieser Sprache \enquote{gehört}. \\
    \vspace*{0.5em}
    Automaten können für ebendieses Problem verwendet werden. Sie entscheiden, ob eine gegebene Eingabe zu einer bestimmten Sprache gehört. \\
    \vspace*{0.5em}
    Außerdem bietet die Sprachentheorie die Möglichkeit, bestimmte Eigenschaften von Sprachen zu beweisen oder zu widerlegen.
  \end{exampleblock}
\end{frame}

\subsection{Definitionen}
\begin{frame}[squeeze]{}{}
  \begin{alertblock}{Frage}
    Was ist eine Formale Sprache?
  \end{alertblock}
  
  \pause
  
  \begin{block}{Definition: Formale Sprache}
    Eine Formale Sprache ist eine Menge von Wörtern über einem Alphabet $\Sigma$. \\
  \end{block}
  
  \pause
  
  \begin{block}{Anmerkung}
    Wir werden Sprachen allgemein mit $L$ bezeichnen. \\
    Für jede Sprache gilt
    \[L \subseteq \Sigma^*,\]
    wobei $\Sigma^*$ die Menge aller Wörter bezeichnet, die aus den gegebenen Buchstaben gebildet werden können. \\
  \end{block}
\end{frame}

\begin{frame}[squeeze]{}{}
  \begin{block}{Definition: Alphabet}
    Ein Alphabet $\Sigma$ ist eine Menge von Zeichen. \\
    Die Zeichen des Alphabets werden auch als Buchstaben bezeichnet.
  \end{block}
  
  \pause
  
  \begin{block}{Definition: Wort}
    Ein Wort über einem Alphabet $\Sigma$ ist eine endliche Folge von Zeichen aus $\Sigma$.
  \end{block}
  
  \pause
  
  \begin{block}{Anmerkung}
    Das Wort, das aus $0$ Zeichen besteht, wird als \enquote{Leeres Wort} $\epsilon$ bezeichnet.
  \end{block}
  
  \pause
  
  \begin{alertblock}{Fragen}
    \textcolor{red}{Mächtigkeit von $\Sigma^*$?} \\
    \pause
    \textcolor{red}{Beweis?}
  \end{alertblock}
\end{frame}

\subsection{Beispiele}
\begin{frame}[squeeze]{}{}
  \begin{block}{Beispiele}
    Sei $\Sigma = \{a, b, c\}$.
    \begin{itemize}
      \item<2-> $\Sigma^1$ bezeichnet die Menge aller Wörter der Länge 1 \\
                $\Sigma^1 = \{a, b, c\}$ \\
      \item<3-> $\Sigma^2$ bezeichnet die Menge aller Wörter der Länge 2 \\
                $\Sigma^2 = \{aa, ab, ac, ba, bb, bc, ca, cb, cc\}$ \\
      \item<4-> $\Sigma^k$ bezeichnet die Menge aller Wörter der Länge $k, k \in \mathbb{N}$ \\
      \item<5-> $\Sigma^0$ bezeichnet die Menge aller Wörter der Länge $0$ \\
                $\Sigma^0 = \{\epsilon\}$
      \item<6-> $\Sigma^+$ bezeichnet die Menge aller nicht-leeren Wörter mit
                \[\Sigma^+ = \underset{k \in \mathbb{N}^+}{\bigcup} \Sigma^k\]
      \item<7-> $\Sigma^*$ bezeichnet die Menge aller Wörter mit \\ 
                \[\Sigma^* = \Sigma^0 \cup \Sigma^+\]
    \end{itemize}
    \vspace*{-1em}
  \end{block}
\end{frame}

\subsection{Konkatenation}
\begin{frame}[<+->][squeeze]{}{}
  \begin{block}{Konkatenation}
    Zweistellige Operation in $\Sigma^*$.
      \[\circ : \Sigma^* \times \Sigma^* \rightarrow \Sigma^*\]
      $(\Sigma^*, \circ, \epsilon)$ ist ein Monoid.
  \end{block}
  
  \begin{block}{Beispiel}
    Sei 
      \[x = \text{abc}, y = \text{def}.\]
    Dann gilt:
      \[x \circ y = \text{abcdef},\]
      \[y \circ x = \text{defabc}.\]
  \end{block}
\end{frame}

\section{Wiederholung}
\subsection{Kategorientheorie}
\begin{frame}[squeeze]{}{}
  \begin{definition}
    Ein Monoid ist ein Tripel $(\mathcal{M}, *, e)$ mit
    \begin{enumerate}
      \item Die Verknüpfung $*$ ist assoziativ
        \[a * (b * c) = (a * b) * c = a * b * c\]
        mit $a, b, c \in \mathcal{M}$
      \item $e$ ist neutrales Element bezüglich $*$
        \[e * a = a = a * e\]
        mit $a \in \mathcal{M}$
    \end{enumerate}
    Ein Monoid ist eine Halbgruppe mit ausgezeichnetem neutralen Element.
  \end{definition}
\end{frame}

\section{Sprachen}
\subsection{Beschreibungsmöglichkeiten}
\begin{frame}[<+->][squeeze]{}
  \begin{alertblock}{Frage}
    Wie lassen sich Sprachen beschreiben?
  \end{alertblock}
  
  \begin{exampleblock}{Antwort}
    \begin{itemize}
      \item<2-> Explizite Aufzählung
        \[\mathcal{L} = \{a, ab, abc\}\]
      \item<3-> Mengendefinition
        \[\mathcal{L} = \{w \in \Sigma^* \;|\; w \text{ endet mit a}\}\]
      \item<4-> Angabe eines akzeptierenden Automaten
      \item<5-> Angabe eines Regulären Ausdrucks
    \end{itemize}
  \end{exampleblock}
\end{frame}

\section{Deterministischer Endlicher Automat}
\subsection{Definition}
\begin{frame}[squeeze]{}
  \begin{definition}
    Ein deterministischer Endlicher Automat ist ein 5-Tupel.
    \[\mathcal{A} = (\mathcal{Q}, \Sigma, \delta, q_0, \mathcal{F})\]
    mit
    \[\begin{array}{l l}
        \mathcal{Q} & \text{endliche, nichtleere Menge von Zust\"anden} \\
        \Sigma      & \text{endliche, nichtleere Menge, das Eingabealphabet mit} \\
                    & \Sigma \cap \mathcal{Q} = \emptyset \\
        \delta & \text{überführungsfunktion mit } \delta : \mathcal{Q} \times \Sigma \rightarrow \mathcal{Q} \\
        q_0         & \text{Startzustand} \\
        \mathcal{F} & \text{Menge von Endzust\"anden mit } \mathcal{F} \subseteq \mathcal{Q} \\
      \end{array}\]
  \end{definition}
\end{frame}

\begin{frame}[squeeze]{}
  \begin{block}{Zustandsmenge $\mathcal{Q}$}
    \begin{itemize}
      \item endlich \\
        Ein Endlicher Automat darf nur aus endlich vielen Zust\"anden bestehen. Er hat keinen Speicher, lediglich über den aktuellen Zustand kann er sich merken, was zuletzt (bzw. bisher) geschehen ist. \\
        W\"are die Menge der Zust\"ande nicht endlich, so k\"onnte kein Computer den Automaten simulieren.
      \item nicht leer \\
        Sonst kann er nichts machen.
    \end{itemize}
  \end{block}
\end{frame}

\begin{frame}[squeeze]{}
  \begin{block}{Eingabealphabet $\Sigma$}
    \begin{itemize}
      \item endlich \\
        W\"are das Eingabealphabet nicht endlich, so w\"are auch die überführungsrelation (bezogen auf den zu ihrer Darstellung ben\"otigten Speicherplatz) unendlich groß und kein Computer k\"onnte den Automaten simulieren.
      \item nicht leer \\
        Sonst k\"onnte die Sprache nur $\epsilon$ enthalten und w\"are uninteressant.
      \item $\Sigma \cap \mathcal{Q} = \emptyset$ \\
        Ein Zustand kann nicht gleichzeitig Eingabezeichen sein.
    \end{itemize}
  \end{block}
\end{frame}

\begin{frame}[<+->][squeeze]{}
  \begin{block}{Überführungsfunktion $\delta$}
    \[\delta : \mathcal{Q} \times \Sigma \rightarrow \mathcal{Q}\]
    \begin{enumerate}
      \item<1-> Der Automat befindet sich stets in einem Zustand
      \item<1-> Er liest ein Zeichen der Eingabe
      \item<1-> Die überführungsfunktion $\delta$ berechnet den daraus resultierenden Nachfolgezustand
    \end{enumerate}
  \end{block}
  
  \begin{block}{Startzustand $q_o$}
    Der Startzustand des Automaten wird mit $q_0$ bezeichnet. \\
    Er kann auch einen beliebigen anderen Namen erhalten und muss gekennzeichnet werden.
  \end{block}
\end{frame}

\begin{frame}[<+->][squeeze]{}
  \begin{block}{Endzustandsmenge $\mathcal{F}$}
    \[\mathcal{F} \subseteq \mathcal{Q}\]
    Die Endzust\"ande haben folgende Bedeutung: \\
    Wenn sich der Automat nach Verarbeiten der gesamten Eingabe (also des gesamten Eingabewortes $w$) in einem Zustand aus $\mathcal{F}$ befindet, dann geh\"ort $w$ zu der vom Automaten akzeptierten Sprache.
  \end{block}
  
  \begin{block}{Akzeptierte Sprache}
    Die von $\mathcal{A}$ akzeptierte Sprache $\mathcal{L}_\mathcal{A}$ ist
    \[\mathcal{L_A} = \{w \in \Sigma^* \;|\; \mathcal{A} \text{ befindet sich nach Berechnung in einem Endzustand.} \}\]
  \end{block}
\end{frame}

\section{Exkurs}
\subsection{Mengenlehre}
\begin{frame}[<+->][squeeze]{}
  \begin{alertblock}{Frage}
    Wann ist eine Menge endlich?
    \[|\mathcal{M}| \overset{?}{<} \infty\]
  \end{alertblock}
  
  \begin{exampleblock}{Antwort}
    Eine Menge ist genau dann endlich, wenn sie keine zu sich selbst gleichm\"achtige, echte Teilmenge enth\"alt.
    \[|\mathcal{M}| < \infty \Leftrightarrow \bigwedge_{\mathcal{N} \subsetneq \mathcal{M}} |\mathcal{N}| < |\mathcal{M}|\]
  \end{exampleblock}
\end{frame}

\section{Deterministischer Endlicher Automat}
\subsection{Definition}
\begin{frame}[<+->][squeeze]{}
  \begin{block}{Erweiterte Überführungsfunktion}
    \[\delta^* : \mathcal{Q} \times \Sigma^* \rightarrow \mathcal{Q}\]
    mit
    \[\delta^*(q_i, w) \leftarrow \left\{
                                    \begin{array}{l l}
                                      q_i & \text{für } w = \epsilon \\
                                      \delta(\delta^*(q_i, u), a) & \text{für } w = u \cdot a
                                    \end{array} \right.\]
  \end{block}
  
  \vspace*{-0.5em}
  
  \begin{block}{Akzeptierte Sprache}
    Sei $\mathcal{A} = (\mathcal{Q}, \Sigma, \delta, q_0, \mathcal{F})$ ein Endlicher Automat und $\delta^*$ die zu $\mathcal{A}$ gehörende erweiterte Überführungsfunktion. \\
    Die von $\mathcal{A}$ akzeptierte Sprache $\mathcal{L}_\mathcal{A}$ ist
    \[\mathcal{L_A} = \{w \in \Sigma^* \;|\; \delta^*(q_0, w) \in \mathcal{F}\}.\]
  \end{block}
  
  \vspace*{-0.5em}
  
  \begin{exampleblock}{Anmerkung}
    $\delta^*$ ist die rekursive Beschreibung der Arbeitsweise eines Endlichen Automaten. Die lässt sich durch Wiederholte Anwendung der Regel für $\delta^*$ verdeutlichen.
  \end{exampleblock}
\end{frame}

\subsection{Die erweiterte Überführungsfunktion $\delta^*$}
\begin{frame}[squeeze]{}
  \vspace*{-0.25em}
  \begin{block}{Arbeitsweise eines DEA - Pseudocode}
    Sei $\mathcal{A} = \left( \mathcal{Q}, \Sigma, \delta, q_0, \mathcal{F} \right)$ ein DEA und $w \in \Sigma^*$ ein Eingabewort. Die Abarbeitung von $w$ durch $\mathcal{A}$ verläuft dann wie folgt: \\
    \vspace*{0.5em}
    \texttt{q = q$_0$} \\
    \texttt{for (i = 0; i < |w|;  ++i)} \\
    \texttt{\ \ q = $\delta$(q, w$_i$)} \\
    \texttt{return q $\in \mathcal{F}$}
  \end{block}
  
  \vspace*{-0.5em}
  \pause
  
  \begin{block}{Äquivalente Beschreibung mit $\delta^*$}
    $\delta^*$ ist die rekursive Beschreibung der Arbeitsweise eines Endlichen Automaten und führt zur selben Folge von Zustandsübergängen. \\
    \vspace*{0.5em}
    Sei wieder $\mathcal{A} = \left( \mathcal{Q}, \Sigma, \delta, q_0, \mathcal{F} \right)$ ein DEA und $w = abc$ eine Eingabe für $\mathcal{A}$:
    \vspace*{-0.25em}
    \begin{eqnarray*}
      \delta^* \left( q_0, abc \right) & = & \delta \left( \delta^* \left( q_0, ab \right), c \right) \\
                                       & = & \delta \left( \delta \left( \delta^* \left( q_0, a \right), b \right), c \right) \\
                                       & = & \delta \left( \delta \left( \delta \left( \delta^* \left( q_0, \epsilon \right), a \right), b \right), c \right) \\
                                       & = & \delta \left( \delta \left( \delta \left( q_0, a \right), b \right), c \right)
    \end{eqnarray*}
  \end{block}
\end{frame}

\section{Nichtdeterministischer Endlicher Automat}
\subsection{Definition}
\begin{frame}[squeeze]{}
  \begin{definition}
    Ein nichtdeterministischer Endlicher Automat ist ein 5-Tupel.
    \[\mathcal{A} = (\mathcal{Q}, \Sigma_\epsilon, \delta, q_0, \mathcal{F})\]
    mit
    \[\begin{array}{l l}
        \mathcal{Q}     & \text{endliche, nichtleere Menge von Zust\"anden} \\
        \Sigma_\epsilon & \text{endliche, nichtleere Menge, das Eingabealphabet mit} \\
                        & \Sigma_\epsilon \cap \mathcal{Q} = \emptyset \\
        \delta          & \text{überführungsrelation mit } \delta \subseteq \mathcal{Q} \times \Sigma_\epsilon \times \mathcal{Q} \\
        q_0             & \text{Startzustand} \\
        \mathcal{F}     & \text{Menge von Endzust\"anden mit } \mathcal{F} \subseteq \mathcal{Q} \\
      \end{array}\]
  \end{definition}
\end{frame}

\begin{frame}[<+->][squeeze]{}
  \begin{block}{$\Sigma_\epsilon$}
    Das Eingabealphabet enth\"alt nun auch das leere Wort
    \[\Sigma_\epsilon \leftarrow \Sigma \cup \{\epsilon\}.\]
    Der Automat kann einen Zustandsübergang durchführen, ohne ein Eingabezeichen zu lesen (bzw. indem er $\epsilon$ liest).
  \end{block}
  
  \begin{block}{$\delta$}
    Ein Zustand kann nun mehrere Folgezustände haben
    \[\delta \subseteq \mathcal{Q} \times \Sigma_\epsilon \times \mathcal{Q}\]
    bzw.
    \[\delta : \mathcal{Q} \times \Sigma_\epsilon \rightarrow \mathcal{P(Q)}.\]
  \end{block}
\end{frame}

\section{Reguläre Ausdrücke}
\begin{frame}[<+->][squeeze]{}
  \begin{block}{Regulärer Ausdruck}
    Ein Regulärer Ausdruck beschreibt eindeutig eine Reguläre Menge. Jede Reguläre Menge kann als Reguläre Sprache aufgefasst werden.
  \end{block}
  
  \begin{block}{Reguläre Ausdrücke über einem Alphabet $\Sigma$ sind}
    \begin{itemize}
      \item $\epsilon$
      \item $\emptyset$
      \item $\textbf{a} \Leftrightarrow a \in \Sigma$
      \item $\mathcal{R} \leftarrow \mathcal{R}_1 \circ \mathcal{R}_2$
      \item $\mathcal{R} \leftarrow \mathcal{R}_1 + \mathcal{R}_2$
      \item $\mathcal{R} \leftarrow \mathcal{R}^+$
      \item $\mathcal{R} \leftarrow \mathcal{R}^* = \mathcal{R}^+ + \epsilon$
    \end{itemize}
  \end{block}
\end{frame}

\subsection{Beispiele}
\begin{frame}[<+->][squeeze]{}
\begin{exampleblock}{Beipsiele für Reguläre Ausdrücke mit $\Sigma = \left\{ 0, 1 \right\}$}
\begin{table}
  \centering
  \vspace*{0.5em}
  \begin{tabular}{|l|l|}
    \hline
    Regulärer Ausdruck & Erzeugte Sprache \\
    \hline
    \hline
    $\mathcal{R}_0 = \textbf{1}$ & $\mathcal{L} \left( \mathcal{R}_0 \right) = \left\{ 1 \right\}$ \\
    \hline
    $\mathcal{R}_1 = \textbf{0} + \textbf{1}$ & $\mathcal{L} \left( \mathcal{R}_1 \right) = \left\{ 0, 1 \right\}$ \\
    \hline
    $\mathcal{R}_2 = \textbf{0} \cdot \textbf{1} \left( = \textbf{01} \right)$ & $\mathcal{L} \left( \mathcal{R}_2 \right) = \left\{ 0, 1 \right\}$ \\
    \hline
    $\mathcal{R}_3 = \textbf{1} \cdot \left( \textbf{0} + \textbf{1} \right)$ & $\mathcal{L} \left( \mathcal{R}_3 \right) = \left\{ 10, 11 \right\}$ \\
    \hline
    $\mathcal{R}_4 = \textbf{0}^+ \cdot \textbf{1}$ & $\mathcal{L} \left( \mathcal{R}_4 \right) = \left\{ 01, 001, 0001, 00001, \ldots \right\}$ \\
    \hline
    $\mathcal{R}_5 = \textbf{1}^*$ & $\mathcal{L} \left( \mathcal{R}_5 \right) = \left\{ \epsilon, 1, 11, 111, 1111, \ldots \right\}$ \\
    \hline
    $\mathcal{R}_6 = \mathcal{R}_1 \cdot \mathcal{R}_3$ & $\mathcal{L} \left( \mathcal{R}_6 \right) = \left\{ 010, 011, 110, 111 \right\}$ \\
    \hline
    $\mathcal{R}_7 = \mathcal{R}_1 + \epsilon$ & $\mathcal{L} \left( \mathcal{R}_7 \right) = \left\{ \epsilon, 0, 1 \right\}$ \\
    \hline
    $\mathcal{R}_8 = \mathcal{R}_6 \cdot \emptyset$ & $\mathcal{L} \left( \mathcal{R}_8 \right) = \emptyset$ \\
    \hline
    $\mathcal{R}_9 = \mathcal{R}_1^+ + \mathcal{R}_7$ & $\mathcal{L} \left( \mathcal{R}_9 \right) = \left\{ \epsilon, 0, 1, 00, 01, 10, 11, \ldots \right\} = \Sigma^*$ \\
    \hline    
  \end{tabular}
  \vspace*{-0.5em}
\end{table}
\end{exampleblock}
\end{frame}

\end{document}