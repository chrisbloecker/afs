\documentclass[]{beamer}
\usepackage[utf8x]{inputenc}
\usepackage{beamerthemeshadow}
\usepackage{ucs}
\usepackage{amsmath}
\usepackage{amsfonts}
\usepackage{amssymb}
\usepackage[ngerman]{babel}
\usepackage{graphicx}
\usepackage{textpos}
\usepackage{csquotes}
\usepackage{url}
\usepackage{color}
\usepackage{stmaryrd}

\usetheme[compress]{Berlin}
\usefonttheme{default}
\useoutertheme{infolines}

\setbeamertemplate{footline}%{infolines theme}
{
\leavevmode%
\hbox{%
\begin{beamercolorbox}[wd=.333333\paperwidth,ht=2.25ex,dp=1ex,center]{author in head/foot}%
\usebeamerfont{author in head/foot}\insertshortauthor%~~(\insertshortinstitute)
\end{beamercolorbox}%
\begin{beamercolorbox}[wd=.333333\paperwidth,ht=2.25ex,dp=1ex,center]{title in head/foot}%
\usebeamerfont{title in head/foot}\insertshorttitle
\end{beamercolorbox}%
\begin{beamercolorbox}[wd=.333333\paperwidth,ht=2.25ex,dp=1ex,center]{date in head/foot}%
\usebeamerfont{date in head/foot}\insertshortdate{}
%\insertframenumber{} / \inserttotalframenumber\hspace*{2ex} % ohne Seitenzahlen
\end{beamercolorbox}}%
\vskip0pt%
}

\setbeamertemplate{frametitle}
{%
\vbox{}\vskip-3.9ex
\begin{beamercolorbox}[wd=\paperwidth,ht=2.0ex,leftskip=0.4cm,dp=0.6ex,]{frametitle}
\usebeamerfont{header_font_title}{\insertframetitle}
\end{beamercolorbox}%
}

% Schusterjunge & Hurenkind
\clubpenalty = 10000
\widowpenalty = 10000
\displaywidowpenalty = 10000
\hbadness = 10000

\author[Christopher Blöcker, B. Sc.]{Christopher Blöcker, B. Sc.\\ inf9900@fh-wedel.de}
\title[AFS Tutorium]{Tutorium\\Automaten und Formale Sprachen}
\date{SS 2012}
\subtitle{Grammatiken}
\subject{Automaten und Formale Sprachen}
\begin{document}

\begin{frame}
  \titlepage
\end{frame}

\section{Grammatiken}
\subsection{Beispiel und Definition}
\begin{frame}[squeeze]{}
  \begin{exampleblock}{Beispiel: Bezeichner in \textsc{Pascal} (BNF)}
    \begin{tiny}
      \begin{tabular}{l c l}
        $<$BEZEICHNER$>$ & ::= & $<$BUCHSTABE$>~(<$BUCHSTABE$>\,|\,<$ZIFFER$>)^*$ \\
        $<$BUCHSTABE$>$  & ::= & \textbf{\hspace*{3pt} 'A' $|$ 'B' $|$ 'C' $|$ 'D' $|$ 'E' $|$ 'F' $|$ 'G' $|$ 'H' $|$ 'I' $|$ 'J' $|$ 'K' $|$ 'L' $|$ 'M' $|$ 'N'} \\
                         &     & \textbf{$|$ 'O' $|$ 'P' $|$ 'Q' $|$ 'R' $|$ 'S' $|$ 'T' $|$ 'U' $|$ 'V' $|$ 'W' $|$ 'X' $|$ 'Y' $|$ 'Z} \\
        $<$ZIFFER$>$     & ::= & \textbf{'0' $|$ '1' $|$ '2' $|$ '3' $|$ '4' $|$ '5' $|$ '6' $|$ '7' $|$ '8' $|$ '9'} \\
      \end{tabular}
    \end{tiny}
  \end{exampleblock}
  
  \pause
  
  \begin{block}{Definition : Grammatik}
    Eine Grammatik ist ein 4-Tupel.
    \[\mathcal{G = \left( N, T, R, S \right)}\]
    mit \\
    \vspace*{1em}
    \begin{tabular}{l l}
       $\mathcal{N}$ & Endliche, nicht-leere Menge der Nicht-Terminale \\
       $\mathcal{T}$ & Endliche, nicht-leere Menge der Terminale \\
                     & $\mathcal{N} \cap \mathcal{T} = \emptyset$ \\
       $\mathcal{R}$ & Menge der Produktionen mit \\ 
                     & $\mathcal{R} \subseteq \left( \mathcal{N} \cup \mathcal{T} \right)^+ \times \left( \mathcal{N} \cup \mathcal{T} \right)^*$ \\
       $\mathcal{S}$ & Das Startsymbol mit $\mathcal{S} \in \mathcal{N}$ \\
    \end{tabular}
  \end{block}
\end{frame}

\subsection{Beispiel}
\begin{frame}[squeeze]{}
  \begin{exampleblock}{$\mathcal{L} = \{w \in \Sigma^* \;|\; w = a^nb^n, n \in \mathbb{N}\}$}
    \[\mathcal{G = (N, T, R, S)}\]
    mit
    \[\begin{tabular}{l c l}
      $\mathcal{N}$ & = & $\{S\}$ \\
      $\mathcal{T}$ & = & $\{a, b\}$ \\
      $\mathcal{R}$ & : & $S \to \epsilon$ \\
                    &   & $S \to aSb$ \\
      $\mathcal{S}$ & = & $S$ \\
      \end{tabular}\]
  \end{exampleblock}
  
  \pause
  
  \begin{block}{Anmerkung}
    Regeln mit gleicher linker Seite können zusammengefasst werden 
    \[S \to \epsilon, ~ S \to aSb ~ \Rightarrow ~ S \to \epsilon \;|\; aSb.\]
  \end{block}
\end{frame}

\subsection{\textsc{Chomsky}-Hierarchie}
\begin{frame}[squeeze]{}
  \begin{block}{Die \textsc{Chomsky}-Hierarchie}
    \textsc{Noam Chomsky} hat den Begriff der \textsc{Chomsky}-Hierarchie geprägt und die folgenden $4$ Sprachklassen definiert:
    \begin{itemize}
      \item Typ-3 : Reguläre Sprachen $\mathcal{L}_3$
      \item Typ-2 : Kontext-Freie Sprachen $\mathcal{L}_2$
      \item Typ-1 : Kontext-Sensitive Sprachen $\mathcal{L}_1$
      \item Typ-0 : Rekursiv Aufzählbare Sprachen $\mathcal{L}_0$
    \end{itemize}
    \vspace*{0.5em}
    \pause
    Es gilt:
    \[\mathcal{L}_3 \subsetneq \mathcal{L}_2 \subsetneq \mathcal{L}_1 \subsetneq \mathcal{L}_0\]
    \pause
    Jede dieser Sprachklassen lässt sich durch Grammatiken des jeweiligen Types erzeugen. Dabei unterliegen die Produktionen gewissen Restriktionen.
  \end{block}
  
  \pause
  
  \begin{alertblock}{Anmerkung}
    Es gibt (überabzählbar viele) Sprachen, die \textbf{nicht einmal} rekursiv aufzählbar sind!
  \end{alertblock}
\end{frame}

\begin{frame}[<+->][squeeze]{}
  \begin{block}{Typ-3 Grammatik (Reguläre Sprachen)}
    Sei 
    \[\mathcal{G = (N, T, R, S)}.\] \\
    $\mathcal{G}$ ist vom Typ-3, wenn die Produktionen folgende Form haben:
    \[\begin{array}{l c l}
        A & \rightarrow & \epsilon \\
        A & \rightarrow & a \\
        A & \rightarrow & aB \\
      \end{array}\]
    mit
    \begin{itemize}
      \item<1-> $A, B \in \mathcal{N}$
      \item<1-> $a \in \mathcal{T}$
    \end{itemize}
    Die hier gezeigte Grammatik ist rechtslinear. \\
    (Der Ableitungsbaum wächst nach rechts.)
  \end{block}
\end{frame}

\begin{frame}[<+->][squeeze]{}
  \begin{block}{Typ-3 Grammatik (Reguläre Sprachen)}
    Zu jeder rechtslinearen Typ-3-Grammatik gibt es eine linkslineare Grammatik, die dieselbe Sprache erzeugt. \\
    Rechts- und linkslineare Regeln dürfen nicht vermischt werden, sonst können auch Sprachen erzeugt werden, die nicht vom Typ-3 sind.
  \end{block}
  
  \begin{exampleblock}{$\mathcal{L} = \{w \in \Sigma^* \;|\; w = a^nb^n, n \in \mathbb{N}\}$}
    \[\mathcal{G = (N, T, R, S)}\]
    mit
    \[\begin{tabular}{l c l}
      $\mathcal{N}$ & = & $\{S\}$ \\
      $\mathcal{T}$ & = & $\{a, b\}$ \\
      $\mathcal{R}$ & : & $S \rightarrow aA \;|\; \epsilon$ \\
                    &   & $A \rightarrow Sb$ \\
      $\mathcal{S}$ & = & $S$ \\
      \end{tabular}\]
  \end{exampleblock}
\end{frame}

\begin{frame}[<+->][squeeze]{}
  \begin{block}{Typ-2 Grammatik (Kontext-freie Sprachen)}
    Sei 
    \[\mathcal{G = (N, T, R, S)}.\] \\
    $\mathcal{G}$ ist vom Typ-2, wenn die Produktionen folgende Form haben:
    \[\begin{array}{l c l}
        A & \rightarrow & w \\
      \end{array}\]
    mit
    \begin{itemize}
      \item<1-> $A \in \mathcal{N}$
      \item<1-> $w \in (\mathcal{N} \cup \mathcal{T})^*$
    \end{itemize}
  \end{block}
\end{frame}

\begin{frame}[<+->][squeeze]{}
  \begin{block}{Typ-1 Grammatik (Kontext-sensitive Sprachen)}
    Sei 
    \[\mathcal{G = (N, T, R, S)}.\] \\
    $\mathcal{G}$ ist vom Typ-1, wenn die Produktionen folgende Form haben:
    \[\begin{array}{l c l}
        uAv & \rightarrow & u \phi v \\
      \end{array}\]
    mit
    \begin{itemize}
      \item<1-> $A \in \mathcal{N}$
      \item<1-> $u, v \in (\mathcal{N} \cup \mathcal{T})^*$
      \item<1-> $\phi \in (\mathcal{N} \cup \mathcal{T})^+$
    \end{itemize}
  \end{block}
\end{frame}

\begin{frame}[<+->][squeeze]{}
  \begin{block}{Typ-0 Grammatik (Rekursiv aufzählbare Sprachen)}
    Sei 
    \[\mathcal{G = (N, T, R, S)}.\] \\
    $\mathcal{G}$ ist vom Typ-0, wenn die Produktionen folgende Form haben:
    \[\begin{array}{l c l}
        u & \rightarrow & v \\
      \end{array}\]
    mit
    \begin{itemize}
      \item<1-> $u \in (\mathcal{N} \cup \mathcal{T})^+$
      \item<1-> $v \in (\mathcal{N} \cup \mathcal{T})^*$
    \end{itemize}
  \end{block}
\end{frame}

\begin{frame}[<+->][squeeze]{}
  \begin{block}{Erzeugte Sprache}
    Sei 
    \[\mathcal{G = (N, T, R, S)}.\]
    Die von $\mathcal{G}$ erzeugte Sprache ist
    \[\mathcal{L(G)} = \{w \in \Sigma^* \;|\; \mathcal{S} \overset{*}{\rightarrow} w\}.\]
    "$\overset{*}{\rightarrow}$" bezeichnet eine Folge von Ableitungsschritten.
  \end{block}
  
  \begin{block}{Definition : Ableitungsschritt}
    Ein Ableitungsschritt ist das Anwenden einer Produktion auf eine Satzform, z.B.:
    \begin{itemize}
      \item<2-> $r_1 : A \rightarrow aBc$
      \item<2-> $w = xyz \underline{A} zyx$
    \end{itemize}
    $w \overset{r_1}{\rightarrow} w'$ mit $w' = xyz\underline{aBc}zyx$.
  \end{block}
\end{frame}


\subsection{Mehrdeutigkeit}
\begin{frame}[squeeze]{}
  \begin{block}{Definition: Mehrdeutigkeit}
    Eine Grammatik ist mehrdeutig, wenn es ein Wort gibt, für das unterschiedliche Ableitungsbäume existieren. \\
    \vspace*{0.5em}
    Ob eine Grammatik mehrdeutig ist, ist \textbf{im Allgemeinen nicht entscheidbar}. Es existiert also kein Algorithmus, der zu jeder Grammatik entscheiden kann, ob sie mehrdeutig ist. \\
    \vspace*{0.5em}
    Dennoch ist es unter Vorgabe einer konkreten Grammatik möglich, zu entschieden, ob diese mehrdeutig ist oder nicht.
  \end{block}
  
  \pause
  
  \begin{block}{Nicht-Determinismus in Grammatiken}
    Grammatiken sind nicht-deterministisch, da die angewendeten Produktionsregeln zufällig ausgewählt werden. \\
    \vspace*{0.5em}
    Soll eine Grammatik deterministisch sein, so darf es immer nur eine Produktionsregel zur Auswahl geben.
  \end{block}
\end{frame}

\subsection{Chomsky Normalform}
\begin{frame}[squeeze]{}
  \vspace*{-0.25em}
  \begin{block}{\textsc{Chomsky} Normalform (CNF)}
    Eine Grammatik $\mathcal{G} = \left( \mathcal{N}, \mathcal{T}, \mathcal{R}, \mathcal{S} \right)$ ist in \textsc{Chomsky} Normalform, wenn alle Produktionen folgende Form haben:
    \vspace*{-0.5em}
    \[\begin{array}{l c l}
        A & \rightarrow & BC \\
        A & \rightarrow & a \\
      \end{array}\] \\
    \vspace*{-0.5em}
    Jede Grammatik vom Typ 2 kann in die \textsc{Chomsky} Normalform überführt werden. Dasselbe gilt wegen $\mathcal{L}_3 \subsetneq \mathcal{L}_2$ auch für Grammatiken vom Typ 3. \\
    \vspace*{0.5em}
    Liegt eine Grammatik in CNF vor, so kann mit Hilfe des CYK-Algorithmus das Wortproblem relativ leicht entschieden werden.
  \end{block}
  
  \pause
  \vspace*{-0.5em}
  
  \begin{block}{Das Wortproblem}
    Das Wortproblem bezeichnet die Aufgabe, von einem gegebenen Wort $w$ zu entscheiden, ob es zu einer Sprache $L$ gehört oder nicht, also
    \vspace*{-0.5em}
    \[w \overset{?}{\in} L.\] \\
    \vspace*{-0.5em}
    Das Wortproblem ist für Sprachen der Typen 3, 2, 1 entscheidbar, für Sprachen vom Typ 0 im Allgemeinen aber nicht.
  \end{block}
\end{frame}

\subsection{Transformation nach CNF}
\begin{frame}[squeeze]{}
  \vspace*{-0.25em}
  \begin{alertblock}{Vereinbarung}
    Im folgenden soll gelten: \\
    \vspace*{0.5em}
    Großbuchstaben $A, B, \ldots$ bezeichnen Nichtterminale mit $A, B, \ldots \in \mathcal{N}$. \\
    \vspace*{0.5em}
    Die Kleinbuchstaben $a, b, c$ bezeichnen Terminale mit $a, b, c \in \mathcal{T}$. \\
    \vspace*{0.5em}
    Die Kleinbuchstaben $u, v$ bezeichnen Satzformen mit $u, v \in \left( \mathcal{N} \cup \mathcal{T} \right)^*$
  \end{alertblock}
  
  \pause
  \vspace*{-0.25em}
  
  \begin{exampleblock}{Hinweis}
    Bei der Transformation einer Grammatik nach CNF sollte in der hier beschriebenen Reihenfolge vorgegangen werden.
    \begin{enumerate}
      \item Elimination von $\epsilon$-Produktionen
      \item Elimination von Einheitsproduktionen
      \item Transformation der Produktionen in korrekte Form
    \end{enumerate}
    Da bei der Elimination von $\epsilon$-Produktionen Einheitsproduktionen entstehen können, sollte dieser Schritt durchgeführt werden, bevor die Einheitsproduktionen eliminiert werden. \\
    \vspace*{0.5em}
    Außerdem: Grammatiken in CNF können nicht das leere Wort erzeugen.
  \end{exampleblock}
\end{frame}

\begin{frame}[squeeze]{}
  \vspace*{-0.25em}
  \begin{block}{Elimination von $\epsilon$-Produktionen}
    \begin{itemize}
      \item Auswählen einer $\epsilon$-Produktion
      \item Anwenden der Produktion in allen anderen Produktionen und Aufnehmen der entstehenden Regeln in die Grammatik
      \item Löschen der ausgewählten $\epsilon$-Produktion
    \end{itemize}
    \[\mathcal{R} \leftarrow \left[ \mathcal{R} \cup \left\{ A \to uv \ | \ \left(A, uBv \right), \left( B, \epsilon\right) \in \mathcal{R} \right\} \right] \setminus \left\{ \left( B, \epsilon \right) \right\} \]
    \pause
    Es können auch Regeln existieren, die mehrfache Vorkommen von Nichtterminalen enthalten, die nach $\epsilon$ abgeleitet werden können. \\
    Dann muss jede mögliche Kombination von $\epsilon$-Ableitungen durchgeführt werden. \pause Aus
    \[\begin{array}{l c l}
        A & \rightarrow & BCDCE \ | \ CC \\
        C & \rightarrow & \epsilon \\
      \end{array}\] \\
    \vspace*{-0.75em}
    wird dann
    \[\begin{array}{l c l}
        A & \rightarrow & BCDCE \ | \ BDCE \ | \ BCDE \ | \ BDE \ | \ CC \ | \ C \ | \ \epsilon. \\
      \end{array}\] \\
    \pause
    \vspace*{-0.5em}
    Vorgriff auf Compilerbau: Ein Nichtterminal $A$ mit $A \overset{\ast}{\to} \epsilon$ wird als \enquote{nullable} bezeichnet.
  \end{block}
\end{frame}

\begin{frame}[squeeze]{}
  \begin{block}{Elimination von Einheitsproduktionen}
    \begin{itemize}
      \item Auswählen einer Einheitsproduktion
      \item Einsetzen aller möglichen rechten Seiten und Aufnehmen der entstehenden Regeln in die Grammatik
      \item Löschen der ausgewählten Einheitsproduktion
    \end{itemize}
    \[\mathcal{R} \leftarrow \left[ \mathcal{R} \cup \left\{ A \to u \ | \ \left(A, B \right), \left( B, u \right) \in \mathcal{R} \right\} \right] \setminus \left\{ \left( A, B \right) \right\} \]
    \pause
    Aus
    \[\begin{array}{l c l}
        A & \rightarrow & B \\
        B & \rightarrow & CD \ | \ EFG \ | \ a \ | \ bc \\
      \end{array}\] \\
    \vspace*{-0.75em}
    wird dann
    \[\begin{array}{l c l}
        \hspace*{0.45em} A & \rightarrow & CD \ | \ EFG \ | \ a \ | \ bc. \\
      \end{array}\] \\
  \end{block}
\end{frame}

\begin{frame}[squeeze]{}
  \begin{block}{Transformation von Produktionen}
    \begin{itemize}
      \item Produktionen der Form $A \to a$ müssen nicht verändert werden
      \pause
      \item Für Produktionen der Form $A \to bc$ werden zwei neue Nichtterminale eingefügt mit
            \[\begin{array}{l c l}
              A & \to & BC \\
              B & \to & b \\
              C & \to & c \\
            \end{array}\]
      \pause
      \item Für Produktionen der Form $A \to uv$ mit $|u| = 1, |v| > 1$ wird eine neue Variable eingefügt mit
            \begin{enumerate}
              \item Wenn $u \in \mathcal{N}$
                    \vspace*{-0.25em}
                    \[\begin{array}{l c l}
                      A & \to & uB \\
                      B & \to & v \\
                    \end{array}\]
              \item Wenn $u \in \mathcal{T}$
                    \vspace*{-0.25em}
                    \[\begin{array}{l c l}
                      A & \to & UB \\
                      B & \to & v \\
                      U & \to & u \\
                    \end{array}\]
            \end{enumerate}
            Die Regel $A \to uv$ wird dann gelöscht.
    \end{itemize}
  \end{block}
\end{frame}

\subsection{Feinere Unterteilung der Sprachklassen}
\begin{frame}[squeeze]{}
  \begin{alertblock}{Vereinbarung}
    Im Folgenden bezeichne
    \begin{itemize}
      \item $\mathcal{L}_{i, D}$ die Menge der Deterministischen Typ-i-Sprachen
      \item $\mathcal{L}_{i, N}$ die Menge der Nicht-Deterministischen Typ-i-Sprachen
    \end{itemize}
  \end{alertblock}
  
  \pause
  
  \begin{block}{Reguläre Sprachen (Typ-3)}
    \[\mathcal{L}_{3, D} = \mathcal{L}_{3, N}\]
    Dies ist leicht einzusehen, da zu jedem \textbf{NEA} ein äquivalenter \textbf{DEA} angegeben werden kann.
  \end{block}
\end{frame}

\begin{frame}[<+->][squeeze]{}  
  \begin{block}{Kontext-freie Sprachen (Typ-2)}
    \[\mathcal{L}_{2, D} \subsetneq \mathcal{L}_{2, N}\]
    \textbf{NKA} sind mächtiger als \textbf{DKA}.
  \end{block}
  
  \begin{block}{Kontext-Sensitive Sprachen (Typ-1)}
    \[\mathcal{L}_{1, D} \overset{?}{\subseteq} \mathcal{L}_{1, N}\]
    Es ist bisher nicht bekannt, ob die Inclusion echt oder unecht ist.
  \end{block}
  
  \begin{block}{Rekursiv Aufzählbare Sprachen (Typ-0)}
    \[\mathcal{L}_{0, D} = \mathcal{L}_{0, N}\]
    \textbf{NDTM} können durch \textbf{DTM} simuliert werden, was im Allgemeinen exponentielle Zeit benötigt.
  \end{block}
\end{frame}

\begin{frame}[<+->][squeeze]{}  
  \begin{block}{Rekursiv Aufzählbare Sprachen $\mathcal{L}_{re}$}
    Sei $\mathcal{L} \subseteq \Sigma^*, x \in \mathcal{L}, \mathcal{M}$ eine \textbf{TM} mit $\mathcal{L_M} = \mathcal{L}$. \\
    $\mathcal{L}$ ist Rekursiv Aufzählbar, wenn gilt
    \[\mathcal{M}(x) = \left\{\begin{array}{l l}
                                1        & $wenn $ x \in \mathcal{L} \\
                                \nearrow & $sonst$ \\
                              \end{array}\right.\]
    \begin{center}
      $\mathcal{M}$ \textbf{erkennt} $\mathcal{L}$.
    \end{center}
  \end{block}
  
  \begin{block}{Rekursive Sprachen $\mathcal{L}_r$}
    Sei $\mathcal{L} \subseteq \Sigma^*, x \in \mathcal{L}, \mathcal{M}$ eine \textbf{TM} mit $\mathcal{L_M} = \mathcal{L}$. \\
    $\mathcal{L}$ ist Rekursiv wenn gilt
    \[\mathcal{M}(x) = \left\{\begin{array}{l l}
                                1 & $wenn $ x \in \mathcal{L} \\
                                0 & $wenn $ x \notin \mathcal{L} \\
                              \end{array}\right.\]
    \begin{center}
      $\mathcal{M}$ \textbf{entscheidet} $\mathcal{L}$.
    \end{center}
  \end{block}
\end{frame}

%\begin{frame}[<+->][squeeze]{}  
%  \begin{block}{Zusammenhang}
%    Es gilt
%    \[\mathcal{L}_3 \subsetneq \mathcal{L}_2 \subsetneq \mathcal{L}_1 \subsetneq \mathcal{L}_0.\]
%    Außerdem gilt
%    \begin{itemize}
%      \item<1-> $\mathcal{L}_{3, D} = \mathcal{L}_{3, N} = \mathcal{L}_3$
%      \item<1-> $\mathcal{L}_{2, D} \subsetneq \mathcal{L}_{2, N} = \mathcal{L}_2$
%      \item<1-> $\mathcal{L}_{1, D} \overset{?}{\subseteq} \mathcal{L}_{1, N} = \mathcal{L}_1$
%      \item<1-> $\mathcal{L}_{0, D} = \mathcal{L}_{0, N} = \mathcal{L}_0$
%      \item<1-> $\mathcal{L}_r \subsetneq \mathcal{L}_{re} = \mathcal{L}_0$
%    \end{itemize}
%    $\Rightarrow$
%    \[\mathcal{L}_{3, D} = \mathcal{L}_{3, N} = \mathcal{L}_3 \subsetneq \mathcal{L}_{2, D} \subsetneq \mathcal{L}_{2, N} = \mathcal{L}_2 \subsetneq \mathcal{L}_{1, D} \overset{?}{\subseteq} \mathcal{L}_{1, N} = \mathcal{L}_1\] \[\mathcal{L}_1 \subsetneq \mathcal{L}_r \subsetneq \mathcal{L}_{0, D} = \mathcal{L}_{0, N} = \mathcal{L}_{re} = \mathcal{L}_0\]
%  \end{block}
%\end{frame}

\end{document}