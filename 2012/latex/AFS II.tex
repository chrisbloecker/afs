\documentclass[]{beamer}
\usepackage[utf8x]{inputenc}
\usepackage{beamerthemeshadow}
\usepackage{ucs}
\usepackage{amsmath}
\usepackage{amsfonts}
\usepackage{amssymb}
\usepackage[ngerman]{babel}
\usepackage{graphicx}
\usepackage{textpos}
\usepackage{csquotes}
\usepackage{url}
\usepackage{color}
\usepackage{stmaryrd}

\usetheme[compress]{Berlin}
\usefonttheme{default}
\useoutertheme{infolines}

\setbeamertemplate{footline}%{infolines theme}
{
\leavevmode%
\hbox{%
\begin{beamercolorbox}[wd=.333333\paperwidth,ht=2.25ex,dp=1ex,center]{author in head/foot}%
\usebeamerfont{author in head/foot}\insertshortauthor%~~(\insertshortinstitute)
\end{beamercolorbox}%
\begin{beamercolorbox}[wd=.333333\paperwidth,ht=2.25ex,dp=1ex,center]{title in head/foot}%
\usebeamerfont{title in head/foot}\insertshorttitle
\end{beamercolorbox}%
\begin{beamercolorbox}[wd=.333333\paperwidth,ht=2.25ex,dp=1ex,center]{date in head/foot}%
\usebeamerfont{date in head/foot}\insertshortdate{}
%\insertframenumber{} / \inserttotalframenumber\hspace*{2ex} % ohne Seitenzahlen
\end{beamercolorbox}}%
\vskip0pt%
}

\setbeamertemplate{frametitle}
{%
\vbox{}\vskip-3.9ex
\begin{beamercolorbox}[wd=\paperwidth,ht=2.0ex,leftskip=0.4cm,dp=0.6ex,]{frametitle}
\usebeamerfont{header_font_title}{\insertframetitle}
\end{beamercolorbox}%
}

% Schusterjunge & Hurenkind
\clubpenalty = 10000
\widowpenalty = 10000
\displaywidowpenalty = 10000
\hbadness = 10000

\author[Christopher Blöcker, B. Sc.]{Christopher Blöcker, B. Sc.\\ inf9900@fh-wedel.de}
\title[AFS Tutorium]{Tutorium\\Automaten und Formale Sprachen}
\date{SS 2012}
\subtitle{Teil II.}
\subject{Automaten und Formale Sprachen}
\begin{document}

\begin{frame}
  \titlepage
\end{frame}

\section{Automaten}
\subsection{Äquivalenz}
\begin{frame}[squeeze]{}
  \begin{definition}
    Zwei Automaten $\mathcal{A}$ und $\mathcal{A'}$ sind äquivalent \enquote{$\simeq$}, wenn sie dieselbe Sprache akzeptieren.
    \[\mathcal{A \simeq A' \Leftrightarrow L_A = L_{A'}},\]
    also
    \[\bigwedge_{w \in \Sigma^*} w \in \mathcal{L_A} \Leftrightarrow w \in \mathcal{L_{A'}}.\]
  \end{definition}
  
  \pause
  
  \begin{exampleblock}{Konstruktion}
    Ein zu einem \textbf{NEA} $\mathcal{A}$ äquivalenter \textbf{DEA} $\mathcal{A'}$ kann durch die \textit{Potenzmengenkonstruktion} bestimmt werden.
  \end{exampleblock}
\end{frame}

\subsection{Minimierung von Automaten}
\begin{frame}[squeeze]{}
  \begin{exampleblock}{Potenzmengenkonstruktion}
    Sei $\mathcal{A}$ ein \textbf{NEA} mit
    \[\mathcal{A} = \left( \mathcal{Q}, \Sigma, \delta, q_0, \mathcal{F} \right).\]
    \pause
    Der \textbf{DEA} $\mathcal{A'}$ mit
    \[\mathcal{A' \simeq A}\]
    und
    \[\mathcal{A'} = \left( \mathcal{Q'}, \Sigma', \delta', q_0', \mathcal{F'} \right)\]
    wird bestimmt durch:
    \pause
    \begin{enumerate}
      \item $\mathcal{Q'} \leftarrow \mathcal{P \left( Q \right)}$
      \pause
      \item $\Sigma' \leftarrow \Sigma \setminus \{\epsilon\}$
      \pause
      \item $\delta' \leftarrow \{ \left( q, a, \epsilon \left( r \right) \right) \; | \; \left( q \in \mathcal{Q'} \right) \wedge \left( a \in \Sigma' \right) \wedge ( r \leftarrow \underset{s \in q}{\bigcup} \delta \left( s, a \right) )\}$
      \pause
      \item $q_0' \leftarrow \epsilon \left( \left\{ q_0 \right\} \right)$
      \pause
      \item $\mathcal{F'} \leftarrow \left\{ q \in \mathcal{Q'} \; | \; q \cap \mathcal{F} \neq \emptyset \right\}$
    \end{enumerate}
    \vspace*{0.5em}
    Dabei bezeichne $\epsilon \left( q \right)$ die Epsilon-Hülle von $q$.
  \end{exampleblock}
\end{frame}

\begin{frame}[squeeze]{}
  \begin{alertblock}{Problem}
    Die Überführungsfunktion $\delta$ eines \textbf{DEA} ist nicht surjektiv. \\
    (Es gibt Zustände, die nicht erreicht werden können.)
  \end{alertblock}
  
  \pause
  
  \begin{exampleblock}{Eliminierung unnützer Zustände}
    Die Markierungen der Transitionen sind nicht relevant, der Graph des \textbf{DEA} wird als einfacher Digraph betrachtet. \\
    \pause
    \vspace*{0.5em}
    Die Menge der erreichbaren Zustände lässt sich durch eine Fixpunktiteration bestimmen.
    \begin{enumerate}
      \item $\mathcal{Q}_0 \leftarrow \{q_0\}$
      \item $\mathcal{Q}_{i+1} \leftarrow \mathcal{Q}_i \cup \mathcal{N(Q}_i)$
    \end{enumerate}
    Dabei bezeichne $\mathcal{N(Q}_i)$ die von $\mathcal{Q}_i$ aus erreichbaren Zustände (die Nachbarn von $\mathcal{Q}_i$). \\
    \vspace*{0.5em}
    Die Iteration wird fortgesetzt bis $\mathcal{Q}_{i+1} = \mathcal{Q}_i$.
  \end{exampleblock}
\end{frame}

\section{Reguläre Ausdrücke}
\subsection{Gesetze}
\begin{frame}[squeeze]{}
  \begin{block}{Assoziativität}
    \begin{itemize}
      \item $\left( \mathcal{R}_1 \circ \mathcal{R}_2 \right) \circ \mathcal{R}_3 = \mathcal{R}_1 \circ \left( \mathcal{R}_2 \circ \mathcal{R}_3 \right)$
      \pause
      \item $\left( \mathcal{R}_1 + \mathcal{R}_2 \right) + \mathcal{R}_3 = \mathcal{R}_1 + \left( \mathcal{R}_2 + \mathcal{R}_3 \right)$
    \end{itemize}
  \end{block}
  
  \pause
  
  \begin{block}{Kommutativität}
    \begin{itemize}
      \item $\mathcal{R}_1 + \mathcal{R}_2 = \mathcal{R}_2 + \mathcal{R}_1$
    \end{itemize}
  \end{block}
  
  \pause
  
  \begin{block}{Distributivität}
    \begin{itemize}
      \item $(\mathcal{R}_1 + \mathcal{R}_2) \circ \mathcal{R}_3 = \mathcal{R}_1 \circ \mathcal{R}_3 + \mathcal{R}_2 \circ \mathcal{R}_3$
    \end{itemize}
  \end{block}
  
  \pause
  
  \begin{block}{Neutrale Elemente}
    \begin{itemize}
      \item $\epsilon \circ \mathcal{R} = \mathcal{R} = \mathcal{R} \circ \epsilon$
      \pause
      \item $\emptyset + \mathcal{R} = \mathcal{R} = \mathcal{R} + \emptyset$
    \end{itemize}
  \end{block}
  
  \pause
  
  \begin{alertblock}{Sonderfall: Leere Menge}
    \begin{itemize}
      \item $\mathcal{R} \circ \emptyset = \emptyset = \emptyset \circ \mathcal{R}$
    \end{itemize}
  \end{alertblock}
\end{frame}

\begin{frame}[squeeze]{}
  \begin{block}{Idempotenz}
    \begin{enumerate}
      \item $\left(\mathcal{R}^*\right)^* = \mathcal{R}^*$
      \pause
      \item $\left(\mathcal{R}^+\right)^* = \mathcal{R}^*$
      \pause
      \item $\left(\mathcal{R}^*\right)^+ = \mathcal{R}^*$
      \pause
      \item $\epsilon^* = \epsilon^+ = \epsilon$
      \pause
      \item $\epsilon \circ \mathcal{R}^* = \mathcal{R}^* = \mathcal{R}^* \circ \epsilon$
      \pause
      \item $\mathcal{R}^* \circ \mathcal{R}^* = \mathcal{R}^*$
      \pause
      \item $\mathcal{R}^* + \mathcal{R}^* = \mathcal{R}^*$
      \pause
      \item $\mathcal{R}^+ \circ \mathcal{R}^+ = \mathcal{R} \circ \mathcal{R} \circ \mathcal{R}^*$
      \pause
      \item $\mathcal{R}^+ + \mathcal{R}^+ = \mathcal{R}^+$
      \pause
      \item $\mathcal{R}^* \circ \mathcal{R}^+ = \mathcal{R}^+ = \mathcal{R}^+ \circ \mathcal{R}^*$
    \end{enumerate}
  \end{block}
\end{frame}

\section{Chomsky Hierarchie}
\begin{frame}[squeeze]{}
  \begin{block}{Die \textsc{Chomsky}-Hierarchie}
    \textsc{Noam Chomsky} hat den Begriff der \textsc{Chomsky}-Hierarchie geprägt und die folgenden $4$ Sprachklassen definiert:
    \begin{itemize}
      \item Typ-3 : Reguläre Sprachen $\mathcal{L}_3$
      \item Typ-2 : Kontext-Freie Sprachen $\mathcal{L}_2$
      \item Typ-1 : Kontext-Sensitive Sprachen $\mathcal{L}_1$
      \item Typ-0 : Rekursiv Aufzählbare Sprachen $\mathcal{L}_0$
    \end{itemize}
    \vspace*{0.5em}
    \pause
    Es gilt:
    \[\mathcal{L}_3 \subsetneq \mathcal{L}_2 \subsetneq \mathcal{L}_1 \subsetneq \mathcal{L}_0\]
    \pause
    Jede dieser Sprachklassen lässt sich durch Grammatiken des jeweiligen Types erzeugen. Dabei unterliegen die Produktionen gewissen Restriktionen.
  \end{block}
  
  \pause
  
  \begin{alertblock}{Anmerkung}
    Es gibt (überabzählbar viele) Sprachen, die \textbf{nicht einmal} rekursiv aufzählbar sind!
  \end{alertblock}
\end{frame}

\subsection{Reguläre Sprachen}
\begin{frame}[squeeze]{}
  \begin{alertblock}{Frage}
    Wie lässt sich feststellen, ob
      \[\mathcal{L} \in \mathcal{L}_3 \text{?}\]
  \end{alertblock}
  
  \pause
  
  \begin{exampleblock}{Antwort}
    Eine Sprache ist vom Typ-3, wenn sie von einem endlichen Automaten akzeptiert wird. \\
    \vspace*{0.5em}
    \pause
    Da Endliche Automaten, Reguläre Ausdrücke und Typ-3-Grammatiken äquivalent sind (sie lassen sich ineinander überführen) ist für den Nachweis $\mathcal{L} \in \mathcal{L}_3$ ein Endlicher Automat $\mathcal{A}$, ein Regulärer Ausdruck $\mathcal{R}$ oder eine Typ-3-Grammatik $\mathcal{G}$ mit
      \[\mathcal{L_A} = \mathcal{L} \text{ bzw. } \mathcal{L_R} = \mathcal{L} \text{ bzw. } \mathcal{L_G} = \mathcal{L}\]
    anzugeben.
  \end{exampleblock}
\end{frame}

\begin{frame}[squeeze]{}
  \begin{alertblock}{Frage}
    Wie lässt sich feststellen, ob
      \[\mathcal{L} \notin \mathcal{L}_3 \text{?}\]
  \end{alertblock}
  
  \pause
  
  \begin{exampleblock}{Antwort}
    Mit Hilfe des Pumping-Lemmas. \\
    \vspace*{0.5em}
    \pause
    Das Pumping-Lemma gilt für jede reguläre Sprache.
      \[\mathcal{L} \in \mathcal{L}_3 \to \mathrm{PL} \left( \mathcal{L} \right).\]\\
    \vspace*{0.5em}
    \pause
    Wenn die Bedingungen des Pumping-Lemmas nicht erfüllt sind, so kann die betroffene Sprache nicht regulär sein. \\
    \vspace*{0.5em}
    Durch Kontraposition erhält man
      \[\overline{\mathrm{PL}(\mathcal{L})} \to \mathcal{L} \notin \mathcal{L}_3.\]
  \end{exampleblock}
\end{frame}

\begin{frame}[squeeze]{}
  \begin{block}{Das Pumping-Lemma}
    Sei $\mathcal{L} \subseteq \Sigma^*$ eine reguläre Sprache. \\
    Dann gibt es eine natürliche Zahl $p \in \mathbb{N}$ derart, dass sich jedes Wort $x \in \mathcal{L}$ mit $|x| \geq p$ zerlegen lässt in drei Teilworte
    \vspace*{-0.5em}
      \[x = uvw\]
    \vspace*{-0.5em}
    mit
    \begin{enumerate}
      \item $|v| > 0$
      \item $|uv| \leq p$
      \item $\underset{i \in \mathbb{N}}{\bigwedge} uv^iw \in \mathcal{L}$
    \end{enumerate}
  \end{block}
  
  \vspace*{-0.5em}
  \pause
  
  \begin{alertblock}{Vorsicht!}
    Es gilt
    \vspace*{-0.5em}
    \[\mathcal{L} \in \mathcal{L}_3 \rightarrow \mathrm{PL} \left( \mathcal{L} \right).\]\\
    \vspace*{0.5em}
    Die Umkehrung gilt jedoch \textbf{nicht}! Die folgende Aussage ist also \textbf{falsch}:
    \vspace*{-0.5em}
    \[\lightning ~ \mathrm{PL} \left( \mathcal{L} \right) \rightarrow \mathcal{L} \in \mathcal{L}_3. ~ \lightning\]
  \end{alertblock}
\end{frame}

\section{Exkurs}
\subsection{\textsc{Nerode}-Relation}
\begin{frame}[squeeze]{}
  \begin{alertblock}{Problem}
    Es gibt Sprachen mit $\mathcal{L} \notin \mathcal{L}_3$, für die aber das Pumping-Lemma erfüllt ist.
  \end{alertblock}
  
  \pause  
  
  \begin{exampleblock}{\textsc{Nerode}-Relation}
    Seien $x, y \in \Sigma^*, \mathcal{L} \subseteq \Sigma^*, \mathcal{R} \subseteq \Sigma^* \times \Sigma^*$.
    \[\mathcal{R_N} : x \sim y \Leftrightarrow \underset{z \in \Sigma^*}{\bigwedge} (xz \in \mathcal{L}) \Leftrightarrow (yz \in \mathcal{L})\]
  \end{exampleblock}
  
  \pause
  
  \begin{block}{Satz von \textsc{Myhill} und \textsc{Nerode}}
    Wenn der Index der \textsc{Nerode}-Relation endlich ist, also wenn es nur endlich viele Äquivalenzklassen bezüglich $\mathcal{R_N}$ gibt, so gilt $\mathcal{L} \in \mathcal{L}_3$.
  \end{block}
\end{frame}

\section{Minimierung Endlicher Automaten}
\begin{frame}[squeeze]{}
  \begin{alertblock}{Problem}
    Die Darstellung eines Endlichen Automaten ist nicht eindeutig, es gibt (abzählbar) unendlich viele Automaten, die alle dieselbe Sprache akzeptieren. \\
    \vspace*{0.5em}
    Außerdem kann ein Automat Zustände enthalten, die nutzlos sind oder aber welche, die äquivalent zueinander sind.
  \end{alertblock}
  
  \pause
  
  \begin{exampleblock}{Automaten-Minimierung}
    Bei der Minimierung sollen unnütze Zustände entfernt und äquivalente Zustände zu Äquivalenzklassen zusammengefasst werden. \\
    \vspace*{0.5em}
    Ausgehend von einem Automaten $\mathcal{A}$ entsteht der sogenannte Äquivalenzklassenautomat $\mathcal{A}_\sim$ mit $\mathcal{A} \simeq \mathcal{A}_\sim$.
  \end{exampleblock}
\end{frame}

\begin{frame}[squeeze]{}
  \begin{exampleblock}{Ablauf der Minimierung}
    Sei $\mathcal{A}$ ein DEA.
    \begin{enumerate}
      \item Entfernen unerreichbarer Zustände
      \item Bilden der Äquivalenzklassen
    \end{enumerate}
  \end{exampleblock}
  
  \pause
  
  \begin{block}{Definition der Äquivalenzklassen}
    \[\mathcal{R} \subseteq \mathcal{Q} \times \mathcal{Q}\]
    mit
    \[\mathcal{R} : q_i \sim q_j \Leftrightarrow \underset{w \in \Sigma^*}{\bigwedge} \left[ \delta^* \left( q_i, w \right) \in \mathcal{F} \right] \Leftrightarrow \left[ \delta^* \left( q_j, w \right) \in \mathcal{F} \right].\]
  \end{block}
  
  \pause
  
  \begin{block}{Anmerkung}
    Die entstehenden Äquivalenzklassen stimmen mit denen der \textsc{Nerode}-Relation überein.
  \end{block}
\end{frame}

\begin{frame}[squeeze]{}
  \begin{block}{\textsc{Nerode}-Relation}
    Seien $x, y \in \Sigma^*, \mathcal{L} \subseteq \Sigma^*, \mathcal{R} \subseteq \Sigma^* \times \Sigma^*$.
    \[\mathcal{R_N} : x \sim y \Leftrightarrow \underset{z \in \Sigma^*}{\bigwedge} (xz \in \mathcal{L}) \Leftrightarrow (yz \in \mathcal{L})\]
  \end{block}

  \begin{block}{Definition der Äquivalenzklassen}
    \[\mathcal{R} \subseteq \mathcal{Q} \times \mathcal{Q}\]
    mit
    \[\mathcal{R} : q_i \sim q_j \Leftrightarrow \underset{w \in \Sigma^*}{\bigwedge} \left[ \delta^* \left( q_i, w \right) \in \mathcal{F} \right] \Leftrightarrow \left[ \delta^* \left( q_j, w \right) \in \mathcal{F} \right].\]
  \end{block}  
\end{frame}

\begin{frame}[squeeze]{}
  \begin{block}{Definition der Äquivalenzklassen}
    \[\mathcal{R} : q_i \sim q_j \Leftrightarrow \underset{w \in \Sigma^*}{\bigwedge} \left[ \delta^* \left( q_i, w \right) \in \mathcal{F} \right] \Leftrightarrow \left[ \delta^* \left( q_j, w \right) \in \mathcal{F} \right].\]
  \end{block}
  
  \pause
  
  \begin{alertblock}{Problem}
    Es muss für alle Wörter aus $\Sigma^*$ geprüft werden, ob die Bedingung erfüllt ist. Das ist aber nicht möglich. \\
    \pause
    \vspace*{0.5em}
    Man verschafft sich dadurch Abhilfe, dass man die negierte Aussage verwendet
    \[\mathcal{R} : q_i \nsim q_j \Leftrightarrow \underset{w \in \Sigma^*}{\bigvee} \left[ \delta^* \left( q_i, w \right) \in \mathcal{F} \right] \wedge \left[ \delta^* \left( q_j, w \right) \notin \mathcal{F} \right].\]
    Ein solches $w$ nennen wir einen Zeugen. Das Auffinden von Zeugen geschieht durch eine Fixpunktiteration mit einem Table-Filling-Verfahren.
  \end{alertblock}
\end{frame}

\begin{frame}[squeeze]{}
  \begin{block}{Auffinden von Zeugen}
    \[\mathcal{R} : q_i \nsim q_j \Leftrightarrow \underset{w \in \Sigma^*}{\bigvee} \left[ \delta^* \left( q_i, w \right) \in \mathcal{F} \right] \wedge \left[ \delta^* \left( q_j, w \right) \notin \mathcal{F} \right].\]

    \begin{itemize}
      \item Erzeuge leere Matrix
      \item Markiere End- und Nichtendzustände als nicht äquivalent
      \item Prüfe jedes (bisher nicht markierte) Zustandspaar mit jedem Eingabe\textbf{zeichen} auf Äquivalenz bis keine Paare mehr markiert werden
      \item Fasse die Zustände der nicht markierten Paare zusammen
    \end{itemize}
    
    Das Überprüfen ganzer Wörter ist aufgrund der Transitivität der Äquivalenzrelation nicht notwendig und wird durch die Fixpunktiteration \enquote{simuliert}.
  \end{block}
\end{frame}

\end{document}