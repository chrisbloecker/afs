\documentclass[]{beamer}
\usepackage[utf8x]{inputenc}
\usepackage{beamerthemeshadow}
\usepackage{ucs}
\usepackage{amsmath}
\usepackage{amsfonts}
\usepackage{amssymb}
\usepackage[ngerman]{babel}
\usepackage{graphicx}
\usepackage{textpos}
\usepackage{csquotes}
\usepackage{url}
\usepackage{color}
\usepackage{stmaryrd}

\usetheme[compress]{Berlin}
\usefonttheme{default}
\useoutertheme{infolines}

\setbeamertemplate{footline}%{infolines theme}
{
\leavevmode%
\hbox{%
\begin{beamercolorbox}[wd=.333333\paperwidth,ht=2.25ex,dp=1ex,center]{author in head/foot}%
\usebeamerfont{author in head/foot}\insertshortauthor%~~(\insertshortinstitute)
\end{beamercolorbox}%
\begin{beamercolorbox}[wd=.333333\paperwidth,ht=2.25ex,dp=1ex,center]{title in head/foot}%
\usebeamerfont{title in head/foot}\insertshorttitle
\end{beamercolorbox}%
\begin{beamercolorbox}[wd=.333333\paperwidth,ht=2.25ex,dp=1ex,center]{date in head/foot}%
\usebeamerfont{date in head/foot}\insertshortdate{}
%\insertframenumber{} / \inserttotalframenumber\hspace*{2ex} % ohne Seitenzahlen
\end{beamercolorbox}}%
\vskip0pt%
}

\setbeamertemplate{frametitle}
{%
\vbox{}\vskip-3.9ex
\begin{beamercolorbox}[wd=\paperwidth,ht=2.0ex,leftskip=0.4cm,dp=0.6ex,]{frametitle}
\usebeamerfont{header_font_title}{\insertframetitle}
\end{beamercolorbox}%
}

% Schusterjunge & Hurenkind
\clubpenalty = 10000
\widowpenalty = 10000
\displaywidowpenalty = 10000
\hbadness = 10000

\author[Christopher Blöcker, B. Sc.]{Christopher Blöcker, B. Sc.\\ inf9900@fh-wedel.de}
\title[AFS Tutorium]{Tutorium\\Automaten und Formale Sprachen}
\date{SS 2012}
\subtitle{Teil III.}
\subject{Automaten und Formale Sprachen}
\begin{document}

\begin{frame}
  \titlepage
\end{frame}

\section{Abschlusseigenschaften formaler Sprachen}
\subsection{Reguläre Sprachen}
\begin{frame}[squeeze]{}
  \vspace*{-0.35em}
  \begin{block}{Reguläre Sprachen}
    Die Menge der regulären Sprachen ist abgeschlossen bezüglich
    \begin{itemize}
      \item Komplementbildung
            \vspace*{-0.75em}
            \[\bigwedge_{L \in \mathcal{L}_3} \overline{L} \in \mathcal{L}_3 \hspace*{2.3em}\]
      \item Konkatenation
            \vspace*{-0.75em}
            \[\bigwedge_{L_1, L_2 \in \mathcal{L}_3} L_1 \circ L_2 \in \mathcal{L}_3\]
      \item Vereinigung
            \vspace*{-0.75em}
            \[\bigwedge_{L_1, L_2 \in \mathcal{L}_3} L_1 \cup L_2 \in \mathcal{L}_3\]
      \item Durchschnitt
            \vspace*{-0.75em}
            \[\bigwedge_{L_1, L_2 \in \mathcal{L}_3} L_1 \cap L_2 \in \mathcal{L}_3\]
      \item \textsc{Kleene}'scher Stern
            \vspace*{-0.75em}
            \[\bigwedge_{L \in \mathcal{L}_3} L^* \in \mathcal{L}_3 \hspace*{2.2em}\]
    \end{itemize}
  \end{block}
  
  \pause
  \vspace*{-0.5em}
  
  \begin{exampleblock}{Wie?}
    \pause
    Kombination von Endlichen Automaten, Regulären Ausdrücken, $\ldots$
  \end{exampleblock}
\end{frame}

\subsection{Kontext-freie Sprachen}
\begin{frame}[squeeze]{}
  \begin{block}{Kontext-freie Sprachen}
    Die Menge der kontext-freien Sprachen ist abgeschlossen bezüglich
    \begin{itemize}
      \item Konkatenation
            \[\bigwedge_{L_1, L_2 \in \mathcal{L}_2} L_1 \circ L_2 \in \mathcal{L}_2\]
      \item Vereinigung \\
            \[\bigwedge_{L_1, L_2 \in \mathcal{L}_2} L_1 \cup L_2 \in \mathcal{L}_2\]
      \item \textsc{Kleene}'scher Stern \\
            \[\bigwedge_{L \in \mathcal{L}_2} L^* \in \mathcal{L}_2 \hspace*{2.2em}\]
    \end{itemize}
  \end{block}
  
  \pause
  \vspace*{-0.5em}
  
  \begin{alertblock}{Vorsicht}
    Abgeschlossenheit bezüglich Komplementbildung und Durchschnitt besteht \textbf{nicht}.
  \end{alertblock}
\end{frame}

\subsection{Kontext-sensitive Sprachen}
\begin{frame}[squeeze]{}
  \begin{block}{Kontext-sensitive Sprachen}
    Die Menge der kontext-sensitive Sprachen ist abgeschlossen bezüglich
    \begin{itemize}
      \item Komplementbildung
            \vspace*{-0.25em}
            \[\bigwedge_{L \in \mathcal{L}_1} \overline{L} \in \mathcal{L}_1 \hspace*{2.3em}\]
      \item Konkatenation
            \vspace*{-0.25em}
            \[\bigwedge_{L_1, L_2 \in \mathcal{L}_1} L_1 \circ L_2 \in \mathcal{L}_1\]
      \item Vereinigung
            \vspace*{-0.25em}
            \[\bigwedge_{L_1, L_2 \in \mathcal{L}_1} L_1 \cup L_2 \in \mathcal{L}_1\]
      \item Durchschnitt
            \vspace*{-0.25em}
            \[\bigwedge_{L_1, L_2 \in \mathcal{L}_1} L_1 \cap L_2 \in \mathcal{L}_1\]
      \item \textsc{Kleene}'scher Stern
            \vspace*{-0.25em}
            \[\bigwedge_{L \in \mathcal{L}_1} L^* \in \mathcal{L}_1 \hspace*{2.2em}\]
    \end{itemize}
  \end{block}
\end{frame}

\subsection{Rekursiv aufzählbare Sprachen}
\begin{frame}[squeeze]{}
  \begin{block}{Rekursiv aufzählbare Sprachen}
    Die Menge der rekursiv aufzählbaren Sprachen ist abgeschlossen bezüglich
    \begin{itemize}
      \item Konkatenation
            \[\bigwedge_{L_1, L_2 \in \mathcal{L}_0} L_1 \circ L_2 \in \mathcal{L}_0\]
      \item Vereinigung
            \[\bigwedge_{L_1, L_2 \in \mathcal{L}_0} L_1 \cup L_2 \in \mathcal{L}_0\]
      \item Durchschnitt
            \[\bigwedge_{L_1, L_2 \in \mathcal{L}_0} L_1 \cap L_2 \in \mathcal{L}_0\]
      \item \textsc{Kleene}'scher Stern
            \[\bigwedge_{L \in \mathcal{L}_0} L^* \in \mathcal{L}_0 \hspace*{2.2em}\]
    \end{itemize}
  \end{block}
\end{frame}

\section{Transformationen}
\begin{frame}[squeeze]{}
  \vspace*{-0.25em}
  \begin{block}{Endlicher Automat $\to$ Typ-3-Grammatik}
    Sei $\mathcal{A} = \left( \mathcal{Q}, \Sigma, \delta, q_0, \mathcal{F} \right)$ ein Endlicher Automat. \\
    \vspace*{0.5em}
    Dann kann eine zu $\mathcal{A}$ äquivalente rechtslineare Typ-3-Grammatik $\mathcal{G}_r = \left( \mathcal{N}_r, \mathcal{T}_r, \mathcal{R}_r, \mathcal{S}_r \right)$ erzeugt werden durch
    \pause
    \vspace*{-0.5em}
    \begin{eqnarray*}
      \mathcal{N}_r & \leftarrow & \left\{ Q_i \ | \ q_i \in \mathcal{Q} \right\} \\
      \pause
      \mathcal{T}_r & \leftarrow & \Sigma \\
      \pause
      \mathcal{R}_r & \leftarrow & \left\{ \left( Q_i, aQ_j \right) \ | \ \left( q_i, a, q_j \right) \in \delta \right\} \cup \left\{ \left( Q_i, \epsilon \right) \ | \ q_i \in \mathcal{F} \right\} \hspace*{7.5em} \\
      \pause
      \mathcal{S}_r & \leftarrow & Q_0 \\
    \end{eqnarray*}
      
    \pause
    \vspace*{-1em}
      
    und die äquivalente linkslineare Grammatik $\mathcal{G}_l = \left( \mathcal{N}_l, \mathcal{T}_l, \mathcal{R}_l, \mathcal{S}_l \right)$ durch
    \pause
    \vspace*{-0.5em}
    \begin{eqnarray*}
      \mathcal{N}_l & \leftarrow & \left\{ Q_i \ | \ q_i \in \mathcal{Q} \right\} \cup \left\{ Q_{start} \right\} \\
      \pause
      \mathcal{T}_l & \leftarrow & \Sigma \\
      \pause
      \mathcal{R}_l & \leftarrow & \left\{ \left( Q_j, Q_ia \right) \ | \ ( q_i, a, q_j ) \in \delta \right\} \cup \left\{ \left( Q_{start}, Q_i \right) \ | \ q_i \in \mathcal{F} \right\} \cup \left\{ \left( Q_0, \epsilon \right) \right\} \\
      \pause
      \mathcal{S}_l & \leftarrow & Q_{start} \\
    \end{eqnarray*}
    \vspace*{-3.5em}
  \end{block}
\end{frame}

\begin{frame}[squeeze]{}
  \begin{block}{Rechtslineare Typ-3-Grammatik $\to$ Endlicher Automat}
    Sei $\mathcal{G} = \left( \mathcal{N}, \mathcal{T}, \mathcal{R}, \mathcal{S} \right)$ eine rechtslineare Typ-3-Grammatik. \\
    \vspace*{0.5em}
    Dann kann daraus ein äquivalenter NEA $\mathcal{A} = \left( \mathcal{Q}, \Sigma, \delta, q_0, \mathcal{F} \right)$ erzeugt werden durch
    \pause
    \begin{eqnarray*}
      \mathcal{Q} & \leftarrow & \left\{ q_A \ | \ A \in \mathcal{N} \right\} \cup \left\{ q_{end} \right\} \\
      \pause
      \Sigma & \leftarrow & \mathcal{T} \\
      \pause
      \delta & \leftarrow & \left\{\left( q_A, a, q_B \right) \ | \ \left( A, aB \right) \in \mathcal{R} \right\} \cup \left\{ \left(q_A, a, q_{end} \right) \ | \ \left(A, a \right) \in \mathcal{R} \right\} \\
      \pause
      q_0 & \leftarrow & q_\mathcal{S} \\
      \pause
      \mathcal{F} & \leftarrow & \left\{ q_A \ | \ \left( A, \epsilon \right) \in \mathcal{R} \right\} \cup \left\{ q_{end} \right\} \\
    \end{eqnarray*}
    \\
    \vspace*{-1em}
    Der entstehende Automat ist nicht notwendigerweise minimal und kann unnütze Zustände enthalten.
  \end{block}
\end{frame}

\begin{frame}[squeeze]{}
  \begin{block}{Linkslineare Typ-3-Grammatik $\to$ Endlicher Automat}
    Sei $\mathcal{G} = \left( \mathcal{N}, \mathcal{T}, \mathcal{R}, \mathcal{S} \right)$ eine rechtslineare Typ-3-Grammatik. \\
    \vspace*{0.5em}
    Dann kann daraus ein äquivalenter NEA $\mathcal{A} = \left( \mathcal{Q}, \Sigma, \delta, q_0, \mathcal{F} \right)$ erzeugt werden durch
    \pause
    \begin{eqnarray*}
      \mathcal{Q} & \leftarrow & \left\{ q_A \ | \ A \in \mathcal{N} \right\} \cup \left\{ q_{start} \right\} \\
      \pause
      \Sigma & \leftarrow & \mathcal{T} \\
      \pause
      \delta & \leftarrow & \left\{\left( q_B, a, q_A \right) \ | \ ( A, Ba ) \in \mathcal{R} \right\} \cup \left\{ \left(q_{start}, a, q_A \right) \ | \ \left(A, a \right) \in \mathcal{R} \right\} \\
      \pause
      q_0 & \leftarrow & q_{start} \\
      \pause
      \mathcal{F} & \leftarrow & \left\{ q_\mathcal{S} \right\} \\
    \end{eqnarray*}
    \\
    \vspace*{-1em}
    Der entstehende Automat ist nicht notwendigerweise minimal und kann unnütze Zustände enthalten.
  \end{block}
\end{frame}

\end{document}