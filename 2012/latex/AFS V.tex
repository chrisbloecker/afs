\documentclass[]{beamer}
\usepackage[utf8x]{inputenc}
\usepackage{beamerthemeshadow}
\usepackage{ucs}
\usepackage{amsmath}
\usepackage{amsfonts}
\usepackage{amssymb}
\usepackage[ngerman]{babel}
\usepackage{graphicx}
\usepackage{textpos}
\usepackage{csquotes}
\usepackage{url}
\usepackage{color}
\usepackage{stmaryrd}

\usetheme[compress]{Berlin}
\usefonttheme{default}
\useoutertheme{infolines}

\setbeamertemplate{footline}%{infolines theme}
{
\leavevmode%
\hbox{%
\begin{beamercolorbox}[wd=.333333\paperwidth,ht=2.25ex,dp=1ex,center]{author in head/foot}%
\usebeamerfont{author in head/foot}\insertshortauthor%~~(\insertshortinstitute)
\end{beamercolorbox}%
\begin{beamercolorbox}[wd=.333333\paperwidth,ht=2.25ex,dp=1ex,center]{title in head/foot}%
\usebeamerfont{title in head/foot}\insertshorttitle
\end{beamercolorbox}%
\begin{beamercolorbox}[wd=.333333\paperwidth,ht=2.25ex,dp=1ex,center]{date in head/foot}%
\usebeamerfont{date in head/foot}\insertshortdate{}
%\insertframenumber{} / \inserttotalframenumber\hspace*{2ex} % ohne Seitenzahlen
\end{beamercolorbox}}%
\vskip0pt%
}

\setbeamertemplate{frametitle}
{%
\vbox{}\vskip-3.9ex
\begin{beamercolorbox}[wd=\paperwidth,ht=2.0ex,leftskip=0.4cm,dp=0.6ex,]{frametitle}
\usebeamerfont{header_font_title}{\insertframetitle}
\end{beamercolorbox}%
}

% Schusterjunge & Hurenkind
\clubpenalty = 10000
\widowpenalty = 10000
\displaywidowpenalty = 10000
\hbadness = 10000

\author[Christopher Blöcker, B. Sc.]{Christopher Blöcker, B. Sc.\\ inf9900@fh-wedel.de}
\title[AFS Tutorium]{Tutorium\\Automaten und Formale Sprachen}
\date{SS 2012}
\subtitle{Teil V.}
\subject{Automaten und Formale Sprachen}
\begin{document}

\begin{frame}
  \titlepage
\end{frame}

\section{Aufgaben}
\begin{frame}[squeeze]{}
  \begin{alertblock}{Aufgaben}
    Gegeben sei das Alphabet $\Sigma = \{0, 1\}$. Beweisen Sie:
    \begin{enumerate}
      \item Die Menge aller Typ-0-Sprachen über $\Sigma$ ist abzählbar.
      \item Die Menge aller Sprachen über $\Sigma$ ist überabzählbar.
    \end{enumerate}
  \end{alertblock}
  
  \pause
  
  \begin{exampleblock}{Abzählbarkeit und Überabzählbarkeit}
    \begin{itemize}
      \item Eine Menge $\mathcal{M}$ ist abzählbar, wenn es eine bijektive Funktion $f : \mathcal{M} \to \mathbb{N}$ gibt.
      \item Eine Menge $\mathcal{M}$ ist überabzählbar, wenn sie nicht endlich ist und es keine bijektive Funktion $f : \mathcal{M} \to \mathbb{N}$ gibt.
    \end{itemize}
  \end{exampleblock}
\end{frame}

\subsection{Lösungen}
\begin{frame}[squeeze]{}
  \begin{block}{Abzählbarkeit der Menge der Typ-0-Sprachen über $\Sigma = \{0, 1\}$}
    Typ-0-Sprachen werden von \textbf{TM}'s erkannt, dabei erkennt jede \textbf{TM} genau eine Sprache. \\
    \pause
    \vspace*{0.5em}
    Jeder Turingmaschine kann eine Gödelnummer zugeordnet werden, diese Gödelnummer kann als natürliche Zahl aufgefasst werden. \\
    \pause
    \vspace*{0.5em}
    Somit existiert eine Funktion $f : \textbf{TM} \rightarrow \mathbb{N}$. \\
    \pause
    \vspace*{0.5em}
    $\Rightarrow$ Die Menge der Typ-0-Sprachen muss abzählbar sein. $\hfill \square$
  \end{block}
\end{frame}

\begin{frame}[squeeze]{}
  \begin{block}{Überabzählbarkeit der Menge aller Sprachen über $\Sigma = \{0, 1\}$}
    Der Beweis erfolgt indirekt. \\
    \pause
    \vspace*{0.5em}
    Angenommen, die Menge aller Sprachen über $\Sigma = \{0, 1\}$ sei abzählbar. \\
    Dann lassen sich alle Sprachen in einer zweiseitig unendlichen Matrix aufzählen.
    \pause
    \[\text{Es sei } w_i \in L_j \Leftrightarrow \zeta(i,j) = 1\]
    \[\begin{array}{c||c|c|c|c|c}
        \zeta  & w_1 & w_2 & w_3 & w_4 & \cdots \\
        \hline
        \hline
        L_1    & 1      & 1      & 0      & 1      & \cdots \\
        \hline
        L_2    & 0      & 1      & 0      & 1      & \cdots \\
        \hline
        L_3    & 0      & 0      & 0      & 1      & \cdots \\
        \hline
        L_4    & 1      & 1      & 1      & 0      & \cdots \\
        \hline
        \vdots & \vdots & \vdots & \vdots & \vdots & \ddots \\
      \end{array}\]
      \pause
      Wir konstruieren nun die Sprache $\mathcal{L}$. Dabei soll gelten
      \[w_i \in \mathcal{L} \Leftrightarrow w_i \notin L_i\]
  \end{block}
\end{frame}

\begin{frame}[squeeze]{}
  \begin{block}{Überabzählbarkeit der Menge aller Sprachen über $\Sigma = \{0, 1\}$}
    \[w_i \in \mathcal{L} \Leftrightarrow w_i \notin L_i\]
    \[\begin{array}{c||c|c|c|c|c|c|c}
        \zeta  & w_1        & w_2        & w_3        & w_4        & \cdots \\
        \hline
        \hline
        L_1    & \textbf{1} & 1          & 0          & 1          & \cdots \\
        \hline
        L_2    & 0          & \textbf{1} & 0          & 1          & \cdots \\
        \hline
        L_3    & 0          & 0          & \textbf{0} & 1          & \cdots \\
        \hline
        L_4    & 1          & 1          & 1          & \textbf{0} & \cdots \\
        \hline
        \vdots & \vdots     & \vdots     & \vdots     & \vdots     & \ddots \\
        \hline
        \mathcal{L}    & 0          & 0          & 1          & 1          & \cdots \\
      \end{array}\]
      \pause
      Wir haben nun eine Sprache konstruiert, die von jeder anderen in der Auflistung verschieden sein muss, da es immer wenigstens ein Wort $w_i$ gibt, welches in $\mathcal{L}$ enthalten ist, aber in der jeweiligen Sprache $L_i$ \textbf{nicht}. \\
      \pause
      \vspace*{0.5em}
      Die Annahme, dass die Menge aller Sprachen über $\Sigma = \{0, 1\}$ abzählbar ist, muss somit falsch sein, also ist die Menge aller Sprachen über $\Sigma$ überabzählbar. $\hfill \square$
  \end{block}
\end{frame}

\begin{frame}[squeeze]{}
  \begin{block}{Definition : Deterministische Turingmaschine \textbf{DTM}}
    Eine deterministische Turingmaschine \textbf{DTM} ist ein 7-Tupel.
    \[\mathcal{A} = \left( \mathcal{Q}, \Sigma, \Gamma, \delta, q_0, q_{YES}, q_{NO} \right)\]
    mit
    \[\begin{array}{l l}
        \mathcal{Q} & \text{endliche, nichtleere Menge von Zuständen} \\
        \Sigma      & \text{endliche, nichtleere Menge, das Eingabealphabet mit} \\
                    & \Sigma \cap \mathcal{Q} = \emptyset \text{ und } \sqcup \notin \Sigma \\
        \Gamma      & \text{endliche, nichtleere Menge, das Bandalphabet mit} \\
                    & \Sigma \subset \Gamma \text{ und } \sqcup \in \Gamma \\
        \delta      & \text{Überführungsfunktion mit } \\
                    & \delta : \mathcal{Q} \times \Gamma \rightarrow \mathcal{Q} \times \Gamma \times \{R, M, L\}\\
        q_0         & \text{Startzustand} \\
        q_{YES}     & \text{Akzeptierender Endzustand} \\
        q_{NO}      & \text{Zurückweisender Endzustand}\\
      \end{array}\]
  \end{block}
\end{frame}

\begin{frame}[squeeze]{}
  \begin{block}{Definition : Nichtdeterministische Turingmaschine \textbf{NDTM}}
    Eine nichtdeterministische Turingmaschine \textbf{NDTM} ist ein 7-Tupel.
    \[\mathcal{A} = \left( \mathcal{Q}, \Sigma, \Gamma, \delta, q_0, q_{YES}, q_{NO} \right)\]
    mit
    \[\begin{array}{l l}
        \mathcal{Q} & \text{endliche, nichtleere Menge von Zuständen} \\
        \Sigma      & \text{endliche, nichtleere Menge, das Eingabealphabet mit} \\
                    & \Sigma \cap \mathcal{Q} = \emptyset \text{ und } \sqcup \notin \Sigma \\
        \Gamma      & \text{endliche, nichtleere Menge, das Bandalphabet mit} \\
                    & \Sigma \subset \Gamma \text{ und } \sqcup \in \Gamma \\
        \delta      & \text{Überführungsrelation mit } \\
                    & \delta : \mathcal{Q} \times \Gamma \times \mathcal{Q} \times \Gamma \times \{R, M, L\}\\
        q_0         & \text{Startzustand} \\
        q_{YES}     & \text{Akzeptierender Endzustand} \\
        q_{NO}      & \text{Zurückweisender Endzustand}\\
      \end{array}\]
  \end{block}
\end{frame}

\begin{frame}[squeeze]{}
  \begin{block}{Arbeitsweise einer Turingmaschine}
    \begin{enumerate}
      \item Das Eingabewort $w \in \Sigma^*$ wird in die Bandfelder $0 .. |w-1|$ eingetragen.
      \item Entsprechend der Überführungsfunktion (bzw. -relation) wird der Bandinhalt verändert.
      \item Die Berechnung stoppt, sobald die \textbf{TM} den Zustand $q_{YES}$ oder $q_{NO}$ erreicht hat.
      \item $w \in \mathcal{L(A)}$ wenn die \textbf{TM} in $q_{YES}$ stoppt.
    \end{enumerate}
  \end{block}
  
  \pause
  
  \begin{block}{Definition: Linear beschränkter Automat \textbf{LBA}}
    Ein linear beschränkter Automat ist eine Turingmaschine, bei der das Arbeitsband nur eine endliche Länge hat.
  \end{block}
\end{frame}

\end{document}