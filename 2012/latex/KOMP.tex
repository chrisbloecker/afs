\documentclass[]{beamer}
\usepackage[utf8x]{inputenc}
\usepackage{beamerthemeshadow}
\usepackage{ucs}
\usepackage{amsmath}
\usepackage{amsfonts}
\usepackage{amssymb}
\usepackage[ngerman]{babel}
\usepackage{graphicx}
\usepackage{textpos}
\usepackage{csquotes}
\usepackage{url}
\usepackage{color}
\usepackage{stmaryrd}

\usetheme[compress]{Berlin}
\usefonttheme{default}
\useoutertheme{infolines}

\setbeamertemplate{footline}%{infolines theme}
{
\leavevmode%
\hbox{%
\begin{beamercolorbox}[wd=.333333\paperwidth,ht=2.25ex,dp=1ex,center]{author in head/foot}%
\usebeamerfont{author in head/foot}\insertshortauthor%~~(\insertshortinstitute)
\end{beamercolorbox}%
\begin{beamercolorbox}[wd=.333333\paperwidth,ht=2.25ex,dp=1ex,center]{title in head/foot}%
\usebeamerfont{title in head/foot}\insertshorttitle
\end{beamercolorbox}%
\begin{beamercolorbox}[wd=.333333\paperwidth,ht=2.25ex,dp=1ex,center]{date in head/foot}%
\usebeamerfont{date in head/foot}\insertshortdate{}
%\insertframenumber{} / \inserttotalframenumber\hspace*{2ex} % ohne Seitenzahlen
\end{beamercolorbox}}%
\vskip0pt%
}

\setbeamertemplate{frametitle}
{%
\vbox{}\vskip-3.9ex
\begin{beamercolorbox}[wd=\paperwidth,ht=2.0ex,leftskip=0.4cm,dp=0.6ex,]{frametitle}
\usebeamerfont{header_font_title}{\insertframetitle}
\end{beamercolorbox}%
}

% Schusterjunge & Hurenkind
\clubpenalty = 10000
\widowpenalty = 10000
\displaywidowpenalty = 10000
\hbadness = 10000

\author[Christopher Blöcker, B. Sc.]{Christopher Blöcker, B. Sc.\\ inf9900@fh-wedel.de}
\title[AFS Tutorium]{Tutorium\\Automaten und Formale Sprachen}
\date{SS 2012}
\subtitle{Einführung in die Komplexitätstheorie}
\subject{Automaten und Formale Sprachen}
\begin{document}

\begin{frame}
  \titlepage
\end{frame}

\begin{frame}[squeeze]{}
  \begin{block}{Vereinbarung}
    Im folgenden bezeichne \\
    \vspace*{0.5em}
    \begin{tabular}{l l}
      $\Pi$                            & Entscheidungsprobleme, \\
      $\mathfrak{B} \left( \Pi \right)$ & Menge der Beispiele (Instanzen) von $\Pi$, \\
      $I$                              & Ein Beispiele mit $I \in \mathfrak{B} \left( \Pi \right)$, \\
      $\mathcal{A}_\Pi$                & Einen Algorithmus, der $\Pi$ entscheidet, \\
      $t_\mathcal{A} \left( I \right)$ & Anzahl der Rechenschritte, die $\mathcal{A}$ für das Lösen von $I$ benötigt. \\
    \end{tabular}
  \end{block}
  
  \pause
  
  \begin{definition}
    Die Funktion $T_\mathcal{A} : \mathbb{N} \to \mathbb{N}$ mit
    \[T_\mathcal{A} \left( n \right) = \max \left\{ t_\mathcal{A} \left( I \right) \ | \ I \in \mathfrak{B} \left( \Pi \right) \wedge L \left( I \right) = n \right\} \]
    heißt Zeitkomplexitätsfunktion für den Algorithmus $\mathcal{A}$.
  \end{definition}
\end{frame}

\begin{frame}[squeeze]{}
  \begin{definition}
    Sei $\mathcal{M}$ eine \textbf{DTM}, die Funktion $T_\mathcal{M} : \mathbb{N} \to \mathbb{N}$ mit
    \[T_\mathcal{M} \left( n \right) = \max \{ k \in \mathbb{N} \ | \ \bigvee_{\begin{scriptsize} \begin{array}{c} x \in \Sigma^* \\ |x| = n \end{array} \end{scriptsize}} \mathcal{M} \text{ stoppt nach k Schritten} \} \]
    heißt Zeitkomplexitätsfunktion für die \textbf{TM} $\mathcal{M}$.
  \end{definition}
  
  \pause
  
  \begin{definition}
    Eine \textbf{DTM} heißt Polynomialzeit-\textbf{DTM} (kurz \textbf{PZDTM}), wenn es ein Polynom $p$ gibt mit
    \[\bigwedge_{n \in \mathbb{N}} T_\mathcal{M} \left( n \right) \leq p \left( n \right). \]
  \end{definition}
\end{frame}

\begin{frame}[squeeze]{}
  \begin{block}{$\mathcal{O}$-Notation}
    Es ist im allgemeinen sehr schwierig, eine konkrete Schranke für einen Algorithmus anzugeben, daher begnügt man sich mit ungefähren oberen Schranken.
    \[\begin{array}{l | l}
      \text{Notation}  & \text{Bedeutung} \\
      \hline
      \mathcal{O}(1)   & \text{konstante Laufzeit} \\
      \mathcal{O}(n)   & \text{lineare Laufzeit} \\
      \mathcal{O}(n^k) & \text{polynomiale Laufzeit} \\
      \mathcal{O}(2^n) & \text{exponentielle Laufzeit} \\
    \end{array}\]
  \end{block}
\end{frame}

\begin{frame}[squeeze]{}
  \begin{block}{Definition}
    Die Klasse $\mathcal{P}$ wird definiert als
    \[\mathcal{P} = \left\{L \ | \ L \subseteq \Sigma^* \wedge \text{ es gibt eine } \textbf{PZDTM } \mathcal{M} \text{ mit } L = L_\mathcal{M} \right\}.\]
    $\mathcal{P}$ : Deterministisch in Polynomialzeit lösbar.
  \end{block}
  
  \pause
  
  \begin{exampleblock}{Beispiele für Probleme aus $\mathcal{P}$}
    \begin{itemize}
      \item Sortieren
      \item Suchen
      \item Kürzestes Wege-Problem (Dijkstra)
      \item \textsc{Stcon}
      \item \textsc{Primes}
      \item $\ldots$
    \end{itemize}
  \end{exampleblock}
\end{frame}

\begin{frame}[squeeze]{}
  \begin{definition}
    Die Klasse $\mathcal{NP}$ wird definiert als
    \[\mathcal{NP} = \left\{L \ | \ L \subseteq \Sigma^* \wedge \text{ es gibt eine } \textbf{PZNDTM } \mathcal{M} \text{ mit } L = L_\mathcal{M} \right\}.\]
    $\mathcal{NP}$ : Nichtdeterministisch in Polynomialzeit lösbar. \\
    $\mathcal{NPV}$ bildet dabei die Klasse der schwersten Probleme aus $\mathcal{NP}$.
  \end{definition}
  
  \pause
  
  \begin{exampleblock}{Beispiele für Probleme aus $\mathcal{NPV}$}
    \begin{itemize}
      \item \textsc{Hampath}
      \item \textsc{Traveling Salesman}
      \item \textsc{Vertex Cover}
      \item \textsc{Knapsack}
      \item $\ldots$
    \end{itemize}
  \end{exampleblock}  
\end{frame}

\begin{frame}[squeeze]{}
  \begin{block}{$\mathcal{P}$ und $\mathcal{NP}$}
    Die Probleme aus $\mathcal{P}$ werden als in der Praxis effizient lösbar betrachtet, die aus $\mathcal{NP}$ als nur näherungsweise gut lösbar. \\
    \vspace*{1em}
    Eine mittlerweile weitgehend anerkannte Definition für $\mathcal{NP}$ lautet: \\
    $\mathcal{NP}$ enthält die Probleme, für die eine Lösung in polynomialer Zeit verifiziert werden kann.
  \end{block}
\end{frame}

\begin{frame}[<+->][squeeze]{}
  \begin{block}{Das $\mathcal{P-NP}$-Problem}
    Es gilt
    \[\mathcal{P} \subseteq \mathcal{NP}.\]
    Bis heute konnte jedoch weder
    \[\mathcal{P} = \mathcal{NP}\]
    noch
    \[\mathcal{P} \neq \mathcal{NP}\]
    gezeigt werden. \\
    \vspace*{1em}
    Eine Entscheidung ist derzeit auch nicht in Sicht.
  \end{block}
\end{frame}

\end{document}